% https://www.overleaf.com/learn/latex/Questions/Creating_multiple_bibliographies_in_the_same_document

\documentclass[11pt]{article}
\usepackage{fullpage,url,hyperref}

\def\calM{\mathcal{M}}

\title{Overview of some contributions on computational geometry on various geometric structures beyond the Euclidean structure}

\author{Frank Nielsen}
\date{}
\begin{document}
\maketitle

Let $\calM=\{p_\theta(x), \theta\in\Theta\}$ be a parametric statistical model (regular or not).
Examples of regular statistical models are the families of categorical and Gaussian distributions (exponential families), the families of Gaussian mixture models with  a finite number of prescribed components~\cite{nielsen2018geometry} (mixture families), etc.
Examples of irregular models are families of Gaussian mixture models with $k$ components, the family of uniform distributions, etc.
Information geometry considers various geometric structures on $\calM$. When the statistical model is identifiable, this amounts to the geometry of domains $\Theta$.
I concisely review some geometric structures and geometric computing contributions below.


%%%%%%%%%
\section*{Riemannian geometry}
%%%%%%%%%

The uniqueness and circumcenter of the smallest enclosing ball on a finite point set lying on a Riemannian manifold was studied in~\cite{arnaudon2013approximating}.

%%%%%%%%%
\section*{Finsler geometry}
%%%%%%%%%

Finsler geometry extends Riemannian geometry by considering smoothly varying Minkowski norms at tangent planes of a manifold.
The forward and backward $p$-centers on Finsler manifolds was considered in~\cite{arnaudon2012medians}.


%%%%%%%%%
\section*{Fisher-Rao geometry}
%%%%%%%%%

The Fisher-Rao geometry of a parametric statistical model corresponds to the Riemannian geometry with respect to the Fisher metric.
The Riemannian geodesic distance is called the Fisher-Rao distance in information geometry~\cite{nielsen2020elementary}.
Approximation schemes of the Fisher-Rao distances are considered in~\cite{nielsen2023simple,NIELSEN2024}.
Fisher-Rao geometry of location-scale families amount to hyperbolic geometry.


%%%%%%%%%
\section*{Dually flat geometry}
%%%%%%%%%

Dually flat geometry has the structures of both a Riemannian manifold with a Hessian metric and a pair of dual torsion-free affine connections.
Right-angles in dual geodesic triangles in dually flat spaces are studied in~\cite{nielsen2021geodesic}.
The dual Voronoi diagrams in a dually flat space are dual Bregman Voronoi diagrams~\cite{boissonnat2010bregman} in the dual coordinate systems.
Exact and approximation of the smallest enclosing Bregman balls were studied in~\cite{nock2005fitting,nielsen2008smallest}.
Data structures for proximity queries on dually flat spaces are given in~\cite{nielsen2009tailored,nielsen2009bregman}.
Chernoff information is characterized on a dually flat space in~\cite{nielsen2013information,nielsen2013hypothesis}.
When the dual potential functions are not in closed-form for exponential or mixture families, Monte Carlo information-geometric structures are considered in~\cite{nielsen2019monte}.
When the Bregman generator is separable, the dually flat space amounts to Euclidean geometry~\cite{gomes2018geometry}.





%%%%%%%%%
\section*{Hyperbolic geometry}
%%%%%%%%%
Bisectors in Klein ball model of hyperbolic geometry are affine hyperplanes clipped to the open ball domain~\cite{nielsen2010hyperbolic}.
Thus the Klein hyperbolic Voronoi diagram (HVD) and all its $k$-order Voronoi diagrams are equivalent to  power diagrams clipped to the ball domain.
The geodesics with boundary conditions in Klein model were solved in~\cite{nielsen2012hyperbolic}.
The Klein HVD can be converted to other models of hyperbolic geometry~\cite{nielsen2014visualizing} (demo: HVD \url{https://www.youtube.com/watch?v=i9IUzNxeH4o}, $k$-order HVD \url{https://www.youtube.com/watch?v=sM_16XgyfhY}). 
The hyperbolic smallest enclosing  ball (SEB) in Poincar\'e ball model has an Euclidean shape and thus amounts to an Euclidean smallest enclosing ball.
We can compute numerically the hyperbolic SEB in high dimensions in Klein model with guarantees~\cite{nielsen2015approximating}.
The dual of the HVD is the hyperbolic Delaunay complex~\cite{nielsen2020voronoi}. 
Klein Riemannian geodesics, general position and degeneracies of point sets in hyperbolic geometry are studied in~\cite{nielsen2014further}.
Klein HVDs can be extended to Cayley-Klein HVDs~\cite{nielsen2016classification} where the domains are ellipsoids.
User interfaces based on hyperbolic geometry and information geometry were reported in~\cite{nock2013information}.
Robust embeddings of supervised models in hyperbolic geometry are given in~\cite{neurips-2024}.


%%%%%%%%%
\section*{Hilbert geometry and Birkhoff projective geometry}
%%%%%%%%%

Hilbert geometry is defined on open bounded convex domain. When the domain is a ball, it amounts to Klein model of hyperbolic geometry.
Hilbert geometry of the (a) simplex domain modeling the space of categorical distributions and (b) the elliptope of correlation matrices are studied in~\cite{nielsen2019clustering,nielsen2023non}. 
Balls in Hilbert geometry with polygonal domains are investigated in~\cite{nielsen2017balls}.

For an open bounded convex domain $\Omega$, we may define the cone $C_\Omega=\{(\lambda,\lambda\Omega), \lambda>0\}$ by stacking all its homothets. Birkhoff geometry is a projective geometry which coincides on slices of the cone with the underlying Hilbert geometry~\cite{nielsen2023fisher}.


 
%%%%%%%%%
\section*{Siegel geometry}
%%%%%%%%%

The Siegel upper space is a generalization of the Poincar\'e upper plane: The set of complex square matrices with symmetric positive-definie imaginary parts~\cite{nielsen2013matrix}. The Siegel upper space can be transformed into the Siegel matrix ball which is a generalization of Poincar\'e ball mode of hyperbolic geometry.
The sectional curvatures of the Siegel upper space was shown to be non-positive~\cite{cabanes2021classification}.
The Siegel-Klein geometry~\cite{nielsen2020siegel} is the Hilbert geometry of the Siegel matrix ball model.

%%%%%%%%%
\section*{Regular cone and symmetric cone geometry}
%%%%%%%%%

The cone of symmetric positive-definite matrices is a symmetric cone~\cite{nielsen2023fisher}.
Equivariant log-extrinsic centers and Gaussian-like distributions are studied in~\cite{chevallier2024equivariant}.


%%%%%%%%%
\section*{Bruhat-Tits spaces}
%%%%%%%%%

Bruhat-Tits spaces have a semi parallelogram law.



%%%%%%%%%
\section*{Non-positive-curvature (NPC) spaces}
%%%%%%%%%


%%%%%%%%%
\section*{CAT spaces}
%%%%%%%%%

%%%%%%%%%
\section*{Lightlike manifolds}
%%%%%%%%%

The parameter space of a deep neural network can be considered as a lightlike manifold~\cite{sun2019geometric}.

%%%%%%%%%
\section*{Stratifolds}
%%%%%%%%%

The parameter space of a deep neural network can be considered as a stratifold~\cite{esser2022influence}.

\bibliographystyle{plain}
\bibliography{ContributionsBIB}

\end{document}
 