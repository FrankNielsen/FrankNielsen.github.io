% Frank.Nielsen@acm.org

\documentclass[11pt]{article}
\usepackage{fullpage,amssymb,amsmath,hyperref,url}

\def\eqdef{:=}
\def\eqnota{:=:}
\def\dnu{\mathrm{d}\nu}
\def\calX{\mathcal{X}}
\def\calE{\mathcal{E}}
\def\bbR{\mathbb{R}}
\def\Var{\mathrm{Var}}


\title{Structures cannot be avoided!\\--- Ramsey theory on the intersection graphs of line segments ---}
\date{27 November 2017}
\author{Frank Nielsen}

\begin{document}
\maketitle


This column is also available in pdf: filename \url{RamseyLineSegment.pdf} 
\vskip 0.5cm
 
Consider $n$ line segments $S_1,\ldots, S_n$ on the plane in {\em general position} (ie., segments are either disjoint or intersect in exactly one point).
We ask the following question:\\
How many {\bf pairwise disjoint segments} \underline{OR} how many {\bf pairwise intersecting segments} are there?

Define the {\em intersection graph} $G=(V,E)$ where each segment $S_i$ is associated to a corresponding node $V_i$, and where there is an edge 
$\{V_i,V_j\}$ if and only if the corresponding line segments intersect ($S_i\cap S_j\not=\emptyset$).  
A subset of pairwise intersecting segments corresponds to a clique in $G$, and
a subset of pairwise disjoint segments corresponds to an independent set in $G$ (an anti-clique also called a stable).
Consider the complement graph $\bar{G}=(V,\bar E)$ where $\bar E=\calE\backslash E$, with $\calE=\{\{V_i,V_j\}\ :\ i\not =j\}$ the full edge set.
$\bar G$ is the {\em disjointness graph}~\cite{RamseySegment-2017} of the segments (i.e., an edge between nodes if and only if corresponding segments are disjoint), and we have $G\cup \bar G=K_n$, the clique of size $n$.

Ramsey-type theorems are characterizing the following types of questions:
``How large a structure must  be to guarantee a given property?''
Surprisingly, complete disorder is impossible! 
That is, there always exists  (some) order in structures!

Define the {\bf Ramsey number} $R(s,t)$ as the minimum number $n$ such that any graph with $|V|=n$ nodes
contains either an independent set of size $s$ or a clique $K_t$ of size $t$.

One can prove that those Ramsey numbers are all {\em finite}~\cite{Ramsey-2009} (by proving the recursive formula $R(s,t)\leq R(s-1,t)+R(s,t-1)$ with terminal cases $R(s,1)=R(1,t)=1$ for $s,t\geq 1$), and that the following bound holds (due to the theorem of Erd\"os-Szekeres~\cite{ES-1935}):
$$
R(s,t)\leq \binom{s+t-2}{s-1} <\infty.
$$
 


When $s=t$, $\binom{s+t-2}{s-1}=\binom{2(s-1)}{s-1}$ is a central binomial coefficient that is upper bounded by $2^{2s}$.
Therefore the diagonal Ramsey number $R(s)=R(s,s)$ is upper bounded by $2^{2s}$.
Furthermore, we have the following lower bound: $2^{\frac{s}{2}}< R(s)$ when $s\geq 3$~\cite{ES-1935}.
Thus $s\geq \lfloor \frac{1}{2}\log_2 n \rfloor$.  
Erd\"os proved using a probabilistic argument~\cite{Erdos-1947} that there exists a graph $G$ such that $s\leq 2\log_2 n$ ($G$ and $\bar G$ do not contain $K_s$ subgraphs).
It is proved in~\cite{RamseySegment-2017} (2017) a much stronger result that $s=\Omega(n^{\frac{1}{5}})$ for intersection graphs of line segments: 
Thus there are always $\Omega(n^{\frac{1}{5}})$ pairwise disjoint or pairwise intersecting segments in a set of $n$ segments in general position.

In general, one can consider a {\bf coloring} of the edges of the clique $K_n$ into $c$ colors, and asks for the largest {\bf monochromatic clique} $K_m$ in the edge-colored $K_n$.
For the pairwise disjoint/intersecting line segments, we have $c=2$: 
Say, we color edges red when their corresponding segments intersect and blue, otherwise.


Ramsey's theorem~\cite{Ramsey-2009} (1930) states that for all $c$, 
there exists $n \geq m\geq 2$ such that every $c$-coloring of $K_n$ has a monochromatic clique $K_m$.

Let us conclude with the theorem on acquaintances (people who already met) and strangers (people who meet for the first time):
In a group of six people, either at least three of them are pairwise mutual strangers or at least three of them are pairwise mutual acquaintances.
Consider $K_6$ ($n=6$, 15 edges), and color an edge in red if the edge people  already met and in blue, otherwise.
Then there is a monochromatic triangle ($m=3$). 
Proof: $R(3)=R(3,3)\leq \binom{4}{2}=6$.

Well, it is known that $R(4)=18$ but $R(5)$ is not known! We only know that $43\leq R(5)\leq 48$, $102\leq R(6,6)\leq 165$, etc.
Quantum computers~\cite{QRamsey-2016} can be used to compute Ramsey numbers!


\vskip 1cm
Initially created 27th November 2017 (last updated \today).

\bibliographystyle{plain}
\bibliography{RamseyLineSegment-Bib}
\end{document}


 