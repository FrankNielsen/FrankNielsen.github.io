\documentclass[11pt]{article}
\usepackage{fullpage,amssymb,amsmath,amsthm,url,graphicx}

\def\Cauchy{\mathrm{Cauchy}}

\title{The harmonic mean of two independent Cauchy distributions is a Cauchy distribution}

\author{Frank Nielsen}
\date{December 2021}

\begin{document}

\maketitle

Consider $C_1\sim\Cauchy(l_1,s_1)$ and $C_2\sim\Cauchy(l_2,s_2)$ two independent Cauchy distributions.
Then their harmonic mean 
$$
C_{12}=\frac{1}{\frac{1}{2}\frac{1}{C_1}+\frac{1}{2}\frac{1}{C_2}}=\frac{2\, C_1C_2}{C_1+C_2}
$$
follows a Cauchy distribution.
The proof is based on the following properties of Cauchy distributions:
\begin{itemize}
	\item Let $C\sim\Cauchy(l,s)$ then $\frac{1}{C}\sim\Cauchy\left(\frac{l}{l^2+s^2},\frac{s}{l^2+s^2}\right)$.
	
	\item Let $C\sim\Cauchy(l,s)$ then $\lambda C\sim\Cauchy(\lambda l,\lambda s)$.
	
	\item Let $C_1\sim\Cauchy(l_1,s_1)$ and $C_2\sim\Cauchy(l_2,s_2)$ be two independent Cauchy distributions.
Then $C_1+C_2\sim\Cauchy(l_1+l_2,s_1+s_2)$.
\end{itemize}

It follows that $C_{12}\sim\Cauchy(l_{12},s_{12})$ with

$$
l_{12}=2\frac{(l_1s_2^2+l_2s_1^2+l_1l_2^2+l_1^2l_2)}{(l_1+l_2)^2+(s_1+s_2)^2},
\quad
s_{12}=2\frac{(s_1s_2^2+(s_1^2+l_1^2)s_2+l_2^2s_1)}{(l_1+l_2)^2+(s_1+s_2)^2}.
$$

The following code below in R illustrates the result:
{\small
\begin{verbatim}
# install.packages("univariateML")
library("univariateML")
n <- 100000
l1 <- 1.5
s1 <- 1
l2 <- 2
s2 <- 3
x1 <- rcauchy(n,l1,s1)
x2 <- rcauchy(n,l2,s2)
h12<- 2*x1*x2/(x1+x2)
mlcauchy(h12)
#l12
2*(l1*s2*s2+l2*s1*s1+l1*l2*l2+l1*l1*l2)/((s1+s2)*(s1+s2)+(l1+l2)*(l1+l2))
#s12
2*(s1*s2*s2+(s1*s1+l1*l1)*s2+l2*l2*s1)/((s1+s2)*(s1+s2)+(l1+l2)*(l1+l2))
\end{verbatim}
}


 
%\bibliographystyle{plain}
%\bibliography{NGDBib.bib}

\end{document}
