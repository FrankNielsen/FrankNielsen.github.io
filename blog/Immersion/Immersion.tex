\documentclass[11pt]{article}
\usepackage{fullpage,amssymb,hyperref,url}

\def\eqdef{:=}
\def\eqnota{:=:}
\def\dnu{\mathrm{d}\nu}
\def\calX{\mathcal{X}}
\def\bbR{\mathbb{R}}


\title{Immersion of statistical manifolds\\ Space of spheres}
\author{Frank Nielsen}

\begin{document}
\maketitle

An immersion is a map $i:M\rightarrow N$ from manifold $M$ to manifold $N$ such that
for all $x\in M$, the derivative $d_x i:T_xM\rightarrow T_xN$ is  injective ($\forall a,b\in T_xM, a\not=b\Rightarrow f(a)\not= f(b)$).
That is, an immersion preserves the differential structure.

For example, the Klein bottle can be imbedded in $\bbR^{3}$, but not embedded in $\bbR^{3}$ because of self-intersection.
Whitney theorem states that any $D$-dimensional manifold can be imbedded into the real space $\bbR^{2m-1}$.

In a dually flat space of dimension $D$ with potential function $F(\theta)$, we can improve Whitney theorem by imbedded the statistical manifold
into $\bbR^{R+1}$ with $i(\theta)=(\theta,F(\theta))$, see~\cite{AW-IG-2008}.


Affine immersion~\cite{NS-1994} $(i,v)$

Equiaffine immersion


\vskip 1cm
\date

\bibliographystyle{plain}
\bibliography{ImmersionBib}
\end{document}


