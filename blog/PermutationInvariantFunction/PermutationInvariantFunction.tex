\documentclass{article}
\usepackage{fullpage,amssymb}
\begin{document}

\def\bbR{\mathbb{R}}
\def\bbN{\mathbb{N}}
\def\st{{\ :\ }}

\title{Permutation invariant functions}
\author{Frank Nielsen}
\date{April 2025}
\maketitle

Let $f: 2^X\rightarrow\bbR,  S\mapsto f(S)$ be a real-valued set function where $X=\{x_1,\ldots, x_n\}$ is a set.
It was proven in~\cite{deepset-2017} that $f$ can be written canonically as
\begin{equation}\label{eq:permcandecomp}
f(S)=g\left(\sum_{x\in S} \phi(x)\right)
\end{equation}
for functions $\phi$ and $g$.

First, to prove the sufficient condition, we check that the right-hand-side of Eq.~\ref{eq:permcandecomp} is invariant to any permutation $\sigma$ of the elements of $X$: $f(\sigma(S))=g\left(\sum_{x\in\sigma(S)} \phi(x)\right)=g\left(\sum_{x\in S} \phi(x)\right)$ because of the commutativity property of the addition.
To prove necessity, let $c: X\rightarrow\bbN, x\mapsto c(x)$ be a count function such that $c(x)\not=c(x)\Leftrightarrow x\not=x'$.
Let $\phi(x)$ be any positive function such that $\phi(x)\not=\phi(x')$ for any $x,x'\in X$.
For example, we may choose $\phi(x)=\exp(c(x))$.
Then for any two distinct subsets $S$ and $S'$ of $2^X$, we have $\sum_{x\in\S} \phi(x)\not = \sum_{x\in\S'} \phi(x)$ since the difference is
$$
\sum_{x\in S\Delta S'} \phi(x)>0,
$$
where $S\Delta S'=(S\backslash S')\cup (S'\backslash S)$ denotes the symmetric difference (non-empty since  subsets are distinct).

Thus $\{\phi(S)\st S\in 2^X\}$ is a collection of $|2^X|=2^n$ distinct points in $\bbR$.
We may then choose $g$ to be the Lagrange polynomial interpolating those $2^n$ points $\{(\sum_{x\in\S} \phi(x),f(S))\st S\in 2^X\}$.

For example, the Heron formula for the area of a triangle is invariant to the triangle vertex permutation, and can thus be written using the canonical form of Eq.~\ref{eq:permcandecomp}. See~\cite{DeepSetHeron-2025}.


\bibliographystyle{plain}
\bibliography{permutationfuncBIB}


\end{document}