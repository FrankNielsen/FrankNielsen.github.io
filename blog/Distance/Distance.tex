% Frank.Nielsen@acm.org

\documentclass[11pt]{article}
\usepackage{fullpage,amssymb,amsmath,hyperref,url}

\def\st{\ :\ }
\def\bbF{\mathbb{F}}
\def\eqdef{:=}
\def\eqnota{:=:}
\def\dmu{\mathrm{d}\mu}
\def\dnu{\mathrm{d}\nu}
\def\calX{\mathcal{X}}
\def\calE{\mathcal{E}}
\def\bbR{\mathbb{R}}
\def\Var{\mathrm{Var}}
\def\KL{\mathrm{KL}}
\def\CS{\mathrm{CS}}
\def\calS{\mathcal{S}}
\def\dx{\mathrm{d}x}
\def\calE{\mathcal{E}}
\def\calD{\mathcal{D}}
\def\calX{\mathcal{X}}
\def\calF{\mathcal{F}}
\def\calP{\mathcal{P}}
\def\dmu{\mathrm{d}\mu}
\def\KL{\mathrm{KL}}
\def\JS{\mathrm{JS}}
\def\vJS{\mathrm{vJS}}
\def\IR{\mathrm{IR}}
\def\SME{\mathrm{SME}}
\def\MLE{\mathrm{MLE}}

\sloppy


\title{Discrepancies, dissimilarities, divergences, and distances}

\date{13th August 2021, updated \today}

\author{Frank Nielsen\\ Sony Computer Science Laboratories Inc.\\ Tokyo, Japan}

\begin{document}
\maketitle

This is a working document which will be frequently updated with materials concerning the discrepancy between two distributions.
\vskip 0.5cm
This document is also available in the PDF \url{Distance.pdf} 
\vskip 0.5cm

There are many other acronyms used in the literature for referencing a dissimilarity; For example, the following $5$ D's:
Discrepancies, deviations, dissimilarities, divergences, and distances.

\tableofcontents

%%%
\section{Statistical distances between densities with computationally intractable normalizers}
%%%%

Consider a density $p(x)=\frac{\tilde p(x)}{Z_p}$ where $\tilde p(x)$ is an unnormalized {\em computable} density 
and $Z_p=\int p(x) \dmu(x)$ the {\em computationally intractable} normalizer (also called in statistical physics the partition function or free energy).
A statistical distance $D[p_1:p_2]$ between two densities $p_1(x)=\frac{\tilde p_1(x)}{Z_{p_1}}$ and $p_2(x)=\frac{\tilde p_2(x)}{Z_{p_2}}$ with computationally intractable normalizers $Z_{p_1}$ and $Z_{p_2}$ is said {\em projective} (or two-sided {\em homogeneous}) if and only if
$$
\forall \lambda_1>0,\lambda_2>0,\quad D[p_1:p_2]=D[\lambda_1p_1:\lambda_2 p_2].
$$
In particular, letting $\lambda_1=Z_{p_1}$ and $\lambda_2=Z_{p_2}$, we have
$$
D[p_1:p_2]=D[\tilde{p}_1:\tilde{p}_2].
$$
Notice that the rhs. does not rely on the computationally intractable normalizers.
These projective distances are useful in statistical inference based on minimum distance estimators~\cite{MinDistance-2019} (see next Section).


Here are a few statistical projective distances:

\begin{itemize}
\item {\bf $\gamma$-divergences} ($\gamma>0$)~\cite{gammadivergence-2001,gammadivergence-2008}:
$$
D_{\gamma}[p:q]:=\log \left(\int_{\mathbb{R}} q^{\alpha+1}\right)-\left(1+\frac{1}{\alpha}\right) \log \left(\int_{\mathbb{R}} q^{\alpha} p\right)+\frac{1}{\alpha} \log \left(\int_{\mathbb{R}} p^{\alpha+1}\right),\quad \gamma\geq 0
$$

When $\gamma\rightarrow 0$, we have~\cite{gammadivergence-2008} $D_{\gamma}[p:q]=D_\KL[p:q]$, the Kullback-Leibler divergence (KLD).
For example, we can estimate the KLD between two densities of an exponential-polynomial family by Monte Carlo stochastic integration of the $\gamma$-divergence for a small value of $\gamma$~\cite{PMPEF-2016}.

The $\gamma$-divergences (projective, Bregman-type=Cross-entropy-entropy) and the density power divergence~\cite{BasuPowerDivergence-1998} (non-projective, Bregman-type divergence):
$$
D_{\alpha}^\mathrm{dpd}[p:q]:=\int_{\mathbb{R}} q^{\alpha+1}-\left(1+\frac{1}{\alpha}\right) \int_{\mathbb{R}} q^{\alpha} p+\frac{1}{\alpha} \int_{\mathbb{R}} p^{\alpha+1},\quad \alpha\geq 0,
$$
can be encapsulated into the family of $\Phi$-power divergences~\cite{PhiPowerDivergence-2021} (functional density power divergence class):
$$
D_{\phi, \alpha}[p:q]:=\phi\left(\int_{\mathbb{R}} q^{\alpha+1}\right)-\left(1+\frac{1}{\alpha}\right) \phi\left(\int_{\mathbb{R}} q^{\alpha} p\right)+\frac{1}{\alpha} \phi\left(\int_{\mathbb{R}} p^{\alpha+1}\right),\quad \alpha\geq 0,
$$
where $\phi(e^x)$ convex and strictly increasing, $\phi$ continuous and twice continously differentiable with finite second order derivatives.
We have $D_{\phi,0}[p:q]=\phi'(1)\int_{\mathbb{R}} p(x)\log\frac{p(x)}{q(x)}\dmu(x)=\phi'(1)D_\KL[p:q]$.

\item {\bf Cauchy-Schwarz divergence}~\cite{jenssen2006cauchy} (CSD, projective)
$$
D_\CS[p:q]=-\log \left( \frac{\int p(x) q(x) \dmu(x)}{\sqrt{\int p(x)^{2}  \dmu(x) \int q(x)^{2}  \dmu(x)}} \right) = D_\CS[\lambda_1 p:\lambda_2 q], \forall \lambda_1>0,\lambda_2>0,
$$
and {\bf H\"older divergences}~\cite{HolderDivergence-2017} (HD, projective, which generalizes the CSD):
%$$
%D_{\alpha, \sigma, \tau}^{\mbox{H\"older}}[p:q]=-\log \left(\frac{\int_{\mathcal{X}} p(x)^{\sigma} q(x)^{\tau} \mathrm{d} x}{\left(\int_{\mathcal{X}} p(x)^{\alpha \sigma} \mathrm{d} x\right)^{\frac{1}{\alpha}}\left(\int_{\mathcal{X}} q(x)^{\beta \tau} \mathrm{d} x\right)^{\frac{1}{\beta}}}\right), \quad \frac{1}{\alpha}+\frac{1}{\beta}=1,\alpha\beta>0.
%$$
$$
D_{\alpha, \gamma}^{\mbox{H\"older}}[p:q]=
-\log \left(\frac{\int_{\mathcal{X}} p(x)^{\gamma / \alpha} q(x)^{\gamma / \beta} \mathrm{d} x}{\left(\int_{\mathcal{X}} p(x)^{\gamma} \mathrm{d} x\right)^{1 / \alpha}\left(\int_{\mathcal{X}} q(x)^{\gamma} \mathrm{d} x\right)^{1 / \beta}}\right),\quad \frac{1}{\alpha}+\frac{1}{\beta}=1 .
$$
We have
$$
\forall \lambda_1>0, \lambda_2>0, D_{\alpha, \gamma}^{\mbox{H\"older}}[\lambda_1 p:\lambda_2 q]= D_{\alpha, \gamma}^{\mbox{H\"older}}[p:q],
$$
and
$$
D_{2,2}^{\mbox{H\"older}}[p:q]=D_\CS[p:q].
$$

H\"older divergences between two densities $p_{\theta_p}$ and $p_{\theta_q}$ of an exponential family with cumulant function $F(\theta)$ is available in closed-form~\cite{HolderDivergence-2017}:
$$
D_{\alpha, \gamma}^{\mbox{H\"older}}[p:q]=\frac{1}{\alpha} F\left(\gamma \theta_{p}\right)+\frac{1}{\beta} F\left(\gamma \theta_{q}\right)-F\left(\frac{\gamma}{\alpha} \theta_{p}+\frac{\gamma}{\beta} \theta_{q}\right)
$$

%$$
%D_{\alpha, \sigma, \tau}^{\mbox{H\"older}}[p_{\theta_p}: p_{\theta_q}]=\frac{1}{\alpha} F\left(\alpha \sigma \theta_{p}\right)+\frac{1}{\beta} F\left(\beta \tau \theta_{q}\right)-F\left(\sigma \theta_{p}+\tau \theta_{q}\right).
%$$

The CSD is available in closed-form between mixtures of an exponential family with a conic natural parameter~\cite{nielsen2012closed}: This includes the case of Gaussian mixture models~\cite{kampa2011closed}.
%$$
%D_{\alpha, \sigma, \tau}^{\mathrm{H}}(p: q)=\frac{1}{\alpha} F\left(\alpha \sigma \theta_{p}\right)+\frac{1}{\beta} F\left(\beta \tau \theta_{q}\right)-F\left(\sigma \theta_{p}+\tau \theta_{q}\right).
%$$

\item {\bf Hilbert distance}~\cite{nielsen2019clustering} (projective): Consider two probability mass functions $p=(p_1,\ldots, p_d)$ and $q=(q_1,\ldots,q_d)$ of the $d$-dimensional probability simplex. Then the Hilbert distance is
$$
D^{\mathrm{Hilbert}}[p:q]=\log \left( \frac{\max _{i\in\{1,\ldots, d\}} \frac{p_{i}}{q_{i}}}{\min _{j\in\{1,\ldots, d\}} \frac{p_{j}}{q_{j}}}\right).
$$
We have 
$$
\forall \lambda_1>0, \lambda_2>0, D^{\mbox{Hilbert}}[\lambda_1 p:\lambda_2 q]= D^{\mbox{Hilbert}}[p:q].
$$

The Hilbert projective simplex distance can be extended to the cone of positive-definite matrices~\cite{nielsen2019clustering} (and its subspace of correlation matrices called the elliptope) as follows:
$$
D^{\mathrm{Hilbert}}[P:Q]=\log \left( \frac{\lambda_{\mathrm{max}}(PQ^{-1})}{\lambda_{\mathrm{\min}}(PQ^{-1})} \right),
$$
where $\lambda_{\mathrm{max}}(X)$ and $\lambda_{\mathrm{\min}}(X)$ denote the largest and smallest eigenvalue of matrix $X$, respectively.


\end{itemize}


%%%
\section{Statistical distances between empirical distributions and densities with computationally intractable normalizers}
%%%

When estimating the parameter $\hat\theta$ for a parametric family of distributions $\{p_\theta\}$ from i.i.d. observations $\calS=\{x_1,\ldots,x_n\}$, we can define a minimum distance estimator (MDE):
$$
\hat\theta=\arg\min_\theta D[p_\calS:p_\theta],
$$
where $p_\calS=\frac{1}{n}\sum_{i=1}^n \delta_{x_i}$ is the empirical distribution (normalized).
Thus we need only a right-sided projective divergence to estimate models with computationally intractable normalizers.
For example, the Maximum Likelihood Estimator (MLE) is a MDE wrt. the KLD:
$$
\hat\theta_{\mathrm{MLE}}=\arg\min_\theta D_\KL[p_\calS:p_\theta].
$$
It is thus interesting to study the impact of the choice of the distance $D$ to the properties of the corresponding estimator (e.g., $\gamma$-divergences yields provably robust estimators~\cite{gammadivergence-2008}).



\begin{itemize}
	\item {\bf Hyv\"arinen divergence}~\cite{hyvarinen2005estimation} (also called {\bf Fisher divergence}):
	$$
	D^{\mbox{Hyv\"arinen}}\left[p: p_{\theta}\right]:=\frac{1}{2} \int\left\|\nabla_{x} \log p(x)-\nabla_{x} \log p_{\theta}(x)\right\|^{2}\, p(x) \mathrm{d} x.
	$$
	The Hyvarinen divergence has been extended for order-$\alpha$ Hyvarinen divergences~\cite{nielsen2021fast} (for $\alpha>0$):
	$$
	D^{\mbox{Hyv\"arinen}}_{\alpha}[p: q]:=\frac{1}{2} \int p(x)^{\alpha} \left(\nabla_{x} \log p(x)-\nabla_{x} \log q(x)\right)^{2} \mathrm{d} x, \quad \alpha>0 .
	$$
	
\end{itemize}


\section{The Jensen-Shannon divergence and some generalizations}

%%%
\subsection{Origins of the Jensen-Shannon divergence}
%%%%%

Let $(\calX,\calF,\mu)$ be a measure space, and $(w_1,P_1),\ldots, (w_n,P_n)$ be $n$ weighted probability measures dominated 
by a measure $\mu$ (with $w_i>0$ and $\sum w_i=1$). 
Denote by $\calP:=\{(w_1,p_1),\ldots,  (w_n,p_n)\}$ the set of their weighted Radon-Nikodym densities $p_i=\frac{\mathrm{d}P_i}{\dmu}$ with respect to $\mu$.

A {\em statistical divergence} $D[p:q]$ is a measure of dissimilarity between two densities $p$ and $q$ (i.e., a $2$-point distance) such that $D[p:q]\geq 0$ with equality if and only if $p(x)=q(x)$ $\mu$-almost everywhere.
A {\em statistical diversity index} $D(\calP)$ is a measure of variation of the weighted densities in $\calP$ related to a measure of centrality, i.e., a $n$-point distance which generalizes the notion of $2$-point distance when $\calP_2(p,q):=\{(\frac{1}{2},p_1),(\frac{1}{2},p_2)\}$:
$$
D[p:q]:=D(\calP_2(p,q)).
$$

The fundamental measure of dissimilarity in information theory is the {\em $I$-divergence} (also called the {\em Kullback-Leibler divergence}, KLD,  see Equation~(2.5) page 5~of~\cite{Kullback-1997}):
$$
D_\KL[p:q]:=  \int_\calX p(x)\log\left(\frac{p(x)}{q(x)}\right)\dmu(x).
$$

The KLD is asymmetric (hence the delimiter notation ``:'' instead of `,') but can be symmetrized by defining the Jeffreys {\em $J$-divergence} (Jeffreys divergence, denoted by $I_2$ in Equation (1) in 1946's paper~\cite{Jeffreys-1946}):
$$
D_J[p,q] := D_\KL[p:q]+D_\KL[q:p] = \int_\calX (p(x)-q(x))\log\left(\frac{p(x)}{q(x)}\right)\dmu(x).
$$
Although symmetric, any positive power of Jeffreys divergence fails to satisfy the triangle inequality: 
That is, $D_J^\alpha$ is never a metric distance for any $\alpha>0$, and furthermore $D_J^\alpha$  cannot be upper bounded.

In 1991, Lin proposed the asymmetric {\em $K$-divergence} (Equation (3.2) in~\cite{JS-1991}):
$$
D_K[p:q]:=D_\KL\left[p:\frac{p+q}{2}\right],
$$
and defined the {\em $L$-divergence} by analogy to Jeffreys's symmetrization of the KLD (Equation (3.4) in~\cite{JS-1991}):
$$
D_L[p,q]=D_K[p:q]+D_K[q:p].
$$

By noticing that 
$$
D_L[p,q]= 2 h\left[\frac{p+q}{2}\right]-(h[p]+h[q]),
$$ 
where $h$ denotes Shannon entropy (Equation (3.14) in~\cite{JS-1991}), Lin coined the (skewed) {\em Jensen-Shannon divergence} between two weighted densities $(1-\alpha,p)$ and $(\alpha,q)$ for $\alpha\in(0,1)$ as follows (Equation (4.1) in~\cite{JS-1991}):
\begin{equation}\label{eq:JSh}
D_{\JS,\alpha}[p,q]=h[(1-\alpha)p+\alpha q]-(1-\alpha)h[p]-\alpha h[q].
\end{equation}

Finally, Lin defined the {\em generalized Jensen-Shannon divergence} (Equation (5.1) in~\cite{JS-1991}) for a finite weighted set of densities:
$$
D_\JS[\calP]=h\left[\sum_i w_ip_i\right]-\sum_i w_i h[p_i].
$$
This generalized Jensen-Shannon divergence is nowadays called the {\em Jensen-Shannon diversity index}.

To contrast with the Jeffreys' divergence, the Jensen-Shannon divergence (JSD) $D_\JS:=D_{\JS,\frac{1}{2}}$ is upper bounded by $\log 2$ (does not require the densities to have the same support), and $\sqrt{D_\JS}$ is 
a metric distance~\cite{JSmetric-2003,JSmetric-2004}.
Lin cited precursor work~\cite{WongYOU-1985,JW-1988} yielding definition of the Jensen-Shannon divergence:
The Jensen-Shannon divergence  of Eq.~\ref{eq:JSh} is the so-called ``increments of entropy'' defined in (19) and (20) of~\cite{WongYOU-1985}.

The Jensen-Shannon diversity index was also obtained very differently by Sibson in 1969 when he defined the {\em information radius}~\cite{Sibson-1969} of order $\alpha$ using R\'enyi $\alpha$-means and R\'enyi $\alpha$-entropies~\cite{Renyi-1961}.
In particular, the information radius $\IR_1$ of order $1$ of a weighted set $\calP$ of densities is a diversity index obtained by solving the following variational optimization problem:
\begin{equation}
\IR_{1}[\calP]:=\min_{c} \sum_{i=1}^n w_i D_\KL[p_i:c].  \label{eq:Sibson}
\end{equation}

Sibson solved a more general optimization problem, and obtained the following expression (term $K_1$ in Corollary 2.3~\cite{Sibson-1969}):
$$
\IR_{1}[\calP]=  h\left[\sum_i w_ip_i\right]-\sum_i w_i h[p_i]:=D_\JS[\calP].
$$
Thus Eq.~\ref{eq:Sibson} is a variational definition of the Jensen-Shannon divergence.

%%%
\subsection{Some extensions of the Jensen-Shannon divergence}
%%%%

\begin{itemize}

	\item {\bf Skewing the JSD.} 
	
	The $K$-divergence of Lin can be skewed with a scalar parameter $\alpha\in(0,1)$ to give
	\begin{equation}\label{eq:divK}
	D_{K,\alpha}[p:q]:=D_\KL\left[p:(1-\alpha)p+\alpha q\right].
	\end{equation}
	Skewing parameter $\alpha$ was first studied in~\cite{skewJS-2001} (2001, see Table~2 of~\cite{skewJS-2001}).
	We proposed to unify the Jeffreys divergence with the Jensen-Shannon divergence as follows (Equation 19 in~\cite{nielsen2010family}):
	\begin{equation}\label{eq:JJSalpha}
	D_{K,\alpha}^J[p:q]:=\frac{D_{K,\alpha}[p:q]+D_{K,\alpha}[q:p]}{2}.
	\end{equation}
	When $\alpha=\frac{1}{2}$, we have $D_{K,\frac{1}{2}}^J=D_\JS$, and when $\alpha=1$, we get $D_{K,1}^J=\frac{1}{2}D_J$.
	
	Notice that 
	$$
	D_\JS^{\alpha,\beta}[p;q]:=(1-\beta)D_\KL[p:(1-\alpha)p+\alpha q]+\beta D_\KL[q:(1-\alpha)p+\alpha q]
	$$ 
	amounts to calculate
	 $$
	h^\times[(1-\beta)p+\beta q:(1-\alpha)p+\alpha q]-((1-\beta)h[p]+\beta h[q])
	$$ 
	where 
	$$
	h^\times[p,q]:=\int -p(x)\log q(x)\dmu(x)
	$$ 
	denotes the {\em cross-entropy}. By choosing $\alpha=\beta$, we have $h^\times[(1-\beta)p+\beta q:(1-\alpha)p+\alpha q]=h[(1-\alpha)p+\alpha q]$, 
	and thus recover the skewed Jensen-Shannon divergence of Eq.~\ref{eq:JSh}.
	
	
	In~\cite{nielsen2020generalization} (2020), we considered a positive {\em skewing vector} $\alpha\in [0,1]^k$  and a unit positive weight $w$ belonging to the standard simplex $\Delta_k$, and defined the following {\em vector-skewed Jensen-Shannon divergence}:
\begin{eqnarray}
	D_\JS^{\alpha,w}[p:q] &:=& \sum_{i=1}^k D_\KL[(1-\alpha_i)p_+\alpha_i q : (1-\bar\alpha)p+\bar\alpha q],\\
	&=& h[(1-\bar\alpha)p+\bar\alpha q]-\sum_{i=1}^k h[(1-\alpha_i)p_+\alpha_i q],
\end{eqnarray}
	where $\bar\alpha=\sum_{i=1}^k w_i\alpha_i$. 
	The divergence $D_\JS^{\alpha,w}$ generalizes the (scalar) skew Jensen-Shannon divergence when $k=1$, and is a Ali-Silvey-Csisz\'ar $f$-divergence upper bounded by $\log \frac{1}{\bar\alpha(1-\bar\alpha)}$~\cite{nielsen2020generalization}.
	
	
	\item {\bf A priori mid-density}. The JSD can be interpreted as the total divergence of the densities to the {\em mid-density} $\bar{p}=\sum_{i=1}^n w_i p_i$, a statistical mixture:
	$$
	D_\JS[\calP] = \sum_{i=1}^n w_i D_\KL[p_i:\bar{p}] = h[\bar{p}]-\sum_{i=1}^n w_i h[p_i].
	$$
	Unfortunately, the JSD between two Gaussian densities is not known in closed form because of the definite integral of a log-sum term (i.e., $K$-divergence between a density and a mixture density $\bar{p}$).
	For the special case of the Cauchy family,  a closed-form formula~\cite{CauchyJSD-2021} for the JSD between two Cauchy densities was obtained.
Thus we may choose a {\em geometric mixture distribution}~\cite{JSsym-2019} instead of the ordinary arithmetic mixture $\bar{p}$. More generally, we can choose any weighted mean $M_\alpha$ (say, the geometric mean, or the harmonic mean, or any other power mean) and define a generalization of the $K$-divergence of Equation~\ref{eq:divK}:
	\begin{equation}
	D_K^{M_\alpha}[p:q] := D_K[p:(pq)_{M_\alpha}],
	\end{equation}
	where 
	$$
	(pq)_{M_\alpha}(x):=\frac{M_\alpha(p(x),q(x))}{Z_{M_\alpha}(p:q)}
	$$
	 is a statistical $M$-mixture with $Z_{M_\alpha}(p,q)$
	denoting the normalizing coefficient:
	$$
	Z_{M_\alpha}(p:q)=\int M_\alpha(p(x),q(x))\dmu(x)
	$$ 
	so that $\int (pq)_{M_\alpha}(x)\dmu(x)=1$.
	These $M$-mixtures are well-defined provided the convergence of the definite integrals.
	
	Then we define a generalization of the JSD~\cite{JSsym-2019} termed {\em $(M_\alpha,N_\beta)$-Jensen-Shannon divergence} as follows:
		\begin{equation}
	D_\JS^{M_\alpha,N_\beta}[p:q ]:= N_\beta\left(D_K[p:(pq)_{M_\alpha}] , D_K[q:(pq)_{M_\alpha}]\right),
	\end{equation}
	where $N_\beta$ is yet another weighted mean to average the two $M_\alpha$-$K$-divergences. 
	We have $D_\JS=D_\JS^{A,A}$ where $A(a,b)=\frac{a+b}{2}$ is the arithmetic mean.
	The geometric JSD yields a closed-form formula between two multivariate Gaussians, and has been used in deep learning~\cite{VIGJSD-2020}.
		More generally, we may consider the Jensen-Shannon symmetrization of an arbitrary distance $D$ as  
			\begin{equation}
	D^\JS_{M_\alpha,N_\beta}[p:q]:= N_\beta\left(D[p:(pq)_{M_\alpha}],D[q:(pq)_{M_\alpha}]\right).
	\end{equation}
 %We have the JS-symmetrization of the reverse KLD with respect to the geometric mean which amounts to the skewed Bhattacharyya divergence: $(D_\KL^*)_{G_\alpha}^\JS=D_{\mathrm{Bhat},1-\alpha}$.
	
	\item {\bf A posteriori mid-density}.
	We consider a generalization of Sibson's information radius~\cite{Sibson-1969}.
	Let $S_w(a_1,\ldots,a_n)$ denote a generic weighted mean of $n$ positive scalars $a_1,\ldots, a_n$, with weight vector $w\in\Delta_n$.
	Then we define the {\em $S$-variational Jensen-Shannon diversity index}~\cite{vJSD-2021} as
	\begin{equation}
	D_\vJS^{S_w}(\calP) := \min_{c} S_w\left(D_\KL[p_1:c],D_\KL[p_n:c]\right).
	\end{equation}
	When $S_w=A_w$ (with $A_w(a_1,\ldots,a_n)=\sum_{i=1}^n w_i a_i$ the arithmetic weighted mean), we recover the ordinary Jensen-Shannon diversity index.
			More generally, we define the {\em $S$-Jensen-Shannon index of an arbitrary distance $D$} as
	\begin{equation}
D^\vJS_{S_w}(\calP):=\min_{c} S_w\left(D[p_1:c],\ldots, D[p_n:c]\right).	
\end{equation}
When $n=2$, this yields a Jensen-Shannon-symmetrization of distance $D$.

The variational optimization defining the JSD can also be constrained to a (parametric) family of densities $\calD$, thus defining 
	the {\em $(S,\calD)$-relative Jensen-Shannon diversity index}:
	\begin{equation}
	D_\vJS^{S_w,\calD}(\calP) := \min_{c\in\calD} S_w\left(D_\KL[p_1:c],\ldots, D_\KL[p_n:c]\right).
	\end{equation}
	

The relative Jensen-Shannon divergences are useful for clustering applications:
Let $p_{\theta_1}$ and $p_{\theta_2}$ be two densities of an exponential family $\mathcal{E}$ with cumulant function $F(\theta)$.
Then the $\mathcal{E}$-relative Jensen-Shannon divergence is the Bregman information of $\calP_2(p,q)$ for the conjugate function $F^*(\eta)=-h[p_\theta]$ 
(with $\eta=\nabla F(\theta)$). The $\calE$-relative JSD amounts to  a {\em Jensen divergence} for $F^*$:

\begin{eqnarray}
D_\vJS[p_{\theta_1},p_{\theta_2}] &=& \min_\theta \frac{1}{2}\left\{D_\KL[p_{\theta_1}:p_{\theta}]+D_\KL[p_{\theta_2}:p_{\theta}]\right\},\\
 &=& \min_\theta \frac{1}{2}\left\{B_F[\theta:\theta_1]+B_F[\theta:\theta_2]\right\},\\
 &=&  \min_\eta  \frac{1}{2}\left\{B_{F^*}[\eta_1:\eta]+B_{F^*}[\eta_2:\eta]\right\},\\
&=& \frac{F^*(\eta_1)+F^*(\eta_2)}{2}-F^*(\eta^*),\\
&=:& J_{F^*}(\eta_1,\eta_2),
\end{eqnarray}
since $\eta^*:=\frac{\eta_1+\eta_2}{2}$ (a right-sided {\em Bregman centroid}~\cite{SBD-2009}).
 

	



\end{itemize}

%%%
\section{Statistical distances between mixtures}
%%%%

Pearson~\cite{pearson1894contributions} first considered a unimodal Gaussian mixture of two components for modeling distributions crabs in 1894.
Statistical mixtures~\cite{mclachlan1988mixture} like the Gaussian mixture models (GMMs) are often met in information sciences, and therefore it is important to assess their dissimilarities.
Let $m(x)=\sum_{i=1}^k w_i p_i(x)$ and  $m'(x)=\sum_{i=1}^{k'} w_i' p_i'(x)$ be two finite statistical mixtures.
The KLD between two GMMs $m$ and $m'$ is not analytic~\cite{KLnotanalytic-2004} because of the log-sum terms:
$$
D_\KL[m:m']=\int m(x)\log\frac{m(x)}{m'(x)} \dx.
$$
However, the KLD between two GMMs with the same prescribed components $p_i(x)=p_i'(x)=p_{\mu_i,\Sigma_i}(x)$ (i.e., $k=k'$, and only the normalized positive weights may differ) is provably a Bregman divergence~\cite{wmixtures-2018} for the differential negentropy $F(\theta)$: 
$$
D_\KL[m(\theta):m(\theta')]=B_F(\theta,\theta'),
$$
where $m(\theta)=\sum_{i=1}^{k-1} w_ip_i(x)+(1-\sum_{i=1}^{k-1} w_i)p_k(x)$ and
$F(\theta)=\int m(\theta)\log m(\theta)\dx$. The family $\{m_\theta\: \ \theta\in\Delta_{k-1}^\circ\}$ is called a mixture family in information geometry, where $\Delta_{k-1}^\circ$ denotes the $(k-1)$-dimensional open standard simplex.
However, $F(\theta)$ is usually not available in closed-form because of the log-sum integral.
In some special cases like the mixture of two prescribed Cauchy distributions, we get a closed-form formula for the KLD, JSD, etc.~\cite{CauchyJSD-2021,nielsen2021dually}.
Thus when dealing with mixtures (like GMMs), we either need efficient approximating  (\S\ref{sec:mix:approx}), bounding (\S\ref{sec:mix:bound}) KLD techniques, or new distances (\S\ref{sec:mix:newdist}) that yields closed-form formula between mixture densities.

% , or estimating (\S\ref{sec:mix:est})


%%
\subsection{Approximating and/or fast statistical distances between mixtures}\label{sec:mix:approx}
%%

\begin{itemize}
	\item The Jeffreys divergence (JD) $D_J[m,m']=D_\KL[m:m']+D_\KL[m':m]$ between two (Gaussian) MMs is not available in closed-form, and can be estimated using Monte Carlo integration as 
	$$
\hat{D}_J^{\calS_s}[m,m'] :=   \frac{1}{s} \sum_{i=1}^s 2\frac{(m(x_i)-m'(x_i))}{m(x_i)+m'(x_i)}\log\left(\frac{m(x_i)}{m'(x_i)}\right),
$$
where $\calS_s=\{x_1,\ldots, x_s\}$ are $s$ IID samples from the mid mixture $m_{12}(x):=\frac{1}{2}(m(x)+m'(x))$ (with $\lim_{s\rightarrow \infty} \hat{D}_J^{\calS_s}[m,m']=D_J[m,m']$).
 In~\cite{JeffreysGMMPEF-2021}, the mixtures $m$ and $m'$ are converted into densities of an exponential-polynomial family.
The JD between densities $p_\theta$ and $p_{\theta'}$ of an exponential family with cumulant function $F$ is available in closed-form:
$$
D_J[p_\theta,p_{\theta'}]=(\theta'-\theta)\cdot (\eta'-\eta),
$$ 
with $\eta=\nabla F(\theta)$ and $\theta=\nabla F^*(\eta)$, where $F^*$ denotes the convex conjugate.
Any smooth density $r$ (includes a mixture $r=m$) is converted into  close densities $p_{\theta_r^\MLE}$ and $p^{\eta_r^\SME}$ of a exponential-polynomial family using
extensions of the Maximum Likelihood Estimator (MLE) and Score Matching Estimator (SME).
Then JD between mixtures is approximated as follows
$$
D_J[m,m']\simeq ({\theta'}^\SME-\theta^\SME)\cdot ({\eta'}^\MLE-\eta^\MLE).
$$

\item Given a finite set of mixtures $\{m_i(x)\}$ sharing the same components (e.g., points on a mixture family manifold), we precompute the KLD between  pairwise components to obtain fast approximation of the KLD $D_\KL[m_i:m_j]$ between any two mixtures $m_i$ and $m_j$, see~\cite{Comix-2016}.
 
\end{itemize}





%%
\subsection{Bounding statistical distances between mixtures}\label{sec:mix:bound}
%%

\begin{itemize}

\item {\bf Log-Sum-Exp bounds}: In~\cite{LSE-MM1D-2016,alphadiv-2017}, we lower and upper bound the cross-entropy between mixtures using the fact that the log-sum term $\log m(x)$ and be interpreted as a LSE function. We then compute lower envelopes and upper envelopes of density functions using technique of computational geometry to report deterministic lower and upper bounds on the KLD and $\alpha$-divergences. These bounds are said combinatorial because we decompose the support into elementary intervals. Bounds between the Total Variation Distance (TVD) between univariate mixtures are reported in~\cite{TVmixture-2018}.



\end{itemize}



%\subsection{Estimating statistical distances between mixtures}\label{sec:mix:est}


\subsection{Newly designed statistical distances yielding closed-form formula for mixtures}\label{sec:mix:newdist}


\begin{itemize}
	\item {\bf Statistical Minkowski distances}~\cite{StatMinkGMM-2019}:
	Consider the   Lebesgue space 
	$$
	L_\alpha(\mu)  \eqdef \left\{ f\in \bbF \st  \int_\calX |f(x)|^\alpha \dmu(x) <\infty \right\}
	$$  
	for $\alpha\geq 1$, where   $\bbF$ denotes the set of all real-valued measurable functions defined on the support $\calX$. Minkowski's inequality writes as
$\|p+q\|_\alpha \leq \|p\|_\alpha+\|q\|_\alpha$ for $\alpha\in [1,\infty)$.
 The statistical Minkowski difference distance between $p,q\in L_\alpha(\mu)$ is defined as
\begin{equation}
D_\alpha^{\mathrm{Minkowski}}[p,q] \eqdef \|p\|_\alpha+\|q\|_\alpha - \|p+q\|_\alpha\geq 0.
\end{equation}
The statistical Minkowski log-ratio distance is defined by:
\begin{equation}
L_\alpha^{\mathrm{Minkowski}}[p,q] \eqdef -\log \frac{\|p+q\|_\alpha}{\|p\|_\alpha+\|q\|_\alpha}\geq 0.
\end{equation}
These statistical Minkowski distances are symmetric, and $L_\alpha[p,q]$ is scale-invariant.
For even integers $\alpha\geq 2$, $D_\alpha^{\mathrm{Minkowski}}[m:m']$ is available in closed-form.


\item We show that the Cauchy-Schwarz divergence (CSD), the quadratic Jensen-R\'enyi divergence~\cite{JRGMM-2009} (JRD), and the total square Distance (TSD) between two GMMs, and more generally two mixtures of exponential families, can be obtained in closed-form in~\cite{nielsen2012closed}.
\end{itemize}



%%%
%\section{Distances between exponential family densities}
%%%%
%
%\cite{KLDSigmaPts-2021}
%
%\cite{OnicescuEF-2020}
%
%
%$q$-divergences between $q$-exponential family densities

\vskip 1cm
Initially created 13th August 2021 (last updated \today).

\bibliographystyle{plain}
\bibliography{DistanceBib}
\end{document}
