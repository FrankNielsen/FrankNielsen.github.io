\documentclass{article}
\usepackage{fullpage}

\def\calE{\mathcal{E}}

\def\vectortwotranspose#1#2{{\left[\begin{array}{ll}#1 & #2 \end{array}\right]}}
\def\vectortwo#1#2{{\left[\begin{array}{l}#1 \cr #2 \end{array}\right]}}
\def\matrixtwo#1#2#3#4{{\left[\begin{array}{ll}#1 & #2 \cr  #3 & #4 \end{array}\right]}}

\def\vectorthreetranspose#1#2#3{{\left[\begin{array}{lll}#1 & #2 & #3 \end{array}\right]}}
\def\vectorthree#1#2#3{{\left[\begin{array}{l}#1 \cr #2 \cr #3 \end{array}\right]}}
\def\matrixthree#1#2#3#4#5#6#7#8#9{{\left[\begin{array}{lll}#1 & #2 & #3\cr #4 & #5 & #6\cr #7 & #8 & #9 \end{array}\right]}}

\title{Ellipses}
\author{Frank Nielsen}

\begin{document}
\maketitle

The Cartesian equation of an ellipse $\calE$ is
\begin{equation}\label{eq:ellCC}
ax^2+by^2+cxy+dx+ey+f=0.
\end{equation}

Using the homogeneous coordinates $(x,y,1)$, we can write the above equation in matrix form as
$$
\vectorthreetranspose{x}{y}{1}\, \matrixthree{a}{c/2}{d/2}{c/2}{b}{e/2}{d/2}{e/2}{f} \,  \vectorthree{x}{y}{1}=0.
$$

We can also view an ellipse as an affinely deformed circle.

$$
\calE=\left\{ \vectortwotranspose{x}{y}  + \vectortwo{c_x}{c_y} \right\}
$$ 

\begin{equation}\label{eq:ellCC2}
 ...
\end{equation}


By identifying Eq.~\ref{eq:ellCC} with Eq.~\ref{eq:ellCC2}, we have
$$
...
$$

% https://francisbach.com/matrix-monotony-and-convexity/

The ellipse $\calE$ can also be described using a parametric equation:
\begin{equation}\label{eq:ellCC3}
 ...
\end{equation}


{\tt InEllipse} predicate and 


\section*{Code in {\sc Maxima}}

\end{document}