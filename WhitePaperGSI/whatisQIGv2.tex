\documentclass[a4paper,10pt]{article}
\usepackage[utf8]{inputenc}
\usepackage{amssymb,amsfonts,amsmath,epsfig,float,graphicx,xcolor}

\usepackage[]{hyperref}

\usepackage{mathrsfs}
\usepackage{parskip}


\def\Tr{\mbox{ {\rm Tr}\,}}
\def\Ro{{\mathbb R}}

%opening
\title{What is Quantum Information Geometry?}

\author{Jan Naudts\\Universiteit Antwerpen}

\date{}

\begin{document}

\maketitle


% $<$ INTRO $>$

In Quantum Information Geometry the notion of a statistical manifold is generalized
to that of a quantum statistical manifold.
A related domain of research is that of Quantum Information Theory which concentrates
on the theory behind quantum computing.  

A {\em quantum state} is determined by a wave function $\psi$, which is a normalized element
of a Hilbert space $\mathscr H$. In a statistical context the quantum state is determined
by a {\em density matrix} or a {\em density operator} $\rho$.
This is a positive trace-class operator the trace of which equals 1.
The {\em quantum expectation value} of a bounded operator $B$ on $\mathscr H$
is usually denoted $\langle B\rangle$.
Given an orthonormal diagonalizing basis $(\psi_n)_n$ one can write
\[
 \langle B\rangle=\Tr\rho B=\sum_ip_i (B\psi_n,\psi_n)
\]
where $p_i$ are the eigenvalues of $\rho$. Because the eigenvalues are non-negative
and add up to 1 one can make the interpretation that with probability $p_i$
the quantum system is in the state determined by the wave function $\psi_i$.
The novel aspect of {\em quantum statistics} is that the quantum expectation values 
depend not only on the probabilities $p_i$ but also on the basis of eigenvectors of
the density operator $\rho$. 

The obvious models of Quantum Information Geometry belong to the {\em quantum exponential family},
this is the exponential family of non-degenerate density matrices of dimension $N$-by-$N$. 
See for instance Chapter 7 of \cite{AN00}.
In Statistical Physics the states of a model belonging to the quantum exponential family
are known as quantum Gibbs distributions.
They depend on a number of thermodynamic parameters such as the inverse temperature $\beta$
or a chemical potential $\mu$. 
The importance of these quantum models for different branches of Physics
cannot be overestimated.

A model belonging to the quantum exponential family 
is described by a parameterized density operator $\rho_\theta$, $\theta\in\Ro^n$,
of the form
\[
 \rho_\theta=\exp\big(\theta^iE_i-\alpha(\theta)\big)
\]
with Hermitian $N$-by-$N$ matrices $E_i$ and 
with the normalization function $\phi(\theta)$ given by
\[
 \phi(\theta)=\log\Tr\exp\big(\theta^iE_i\big).
\]
The latter acts as a potential function from which one 
can derive Amari's dually flat geometry \cite{AN00}.
A short calculation gives
\[
\frac{\partial\phi}{\partial\theta^p}=\eta_p
\quad\mbox{ with }\quad \eta_p=\Tr\rho_\theta E_p=\langle E_p\rangle.
\]
These $\eta_p$ are the dual coordinates, dual to the $\theta^p$.

The directional derivatives $\partial\rho_\theta/\partial\theta^i$ of the density matrices
span the tangent spaces of the manifold of quantum states.
Eguchi's method \cite{ES85} can be used to define the inner product
of pairs of tangent vectors starting from Umegaki's relative entropy/divergence \cite{UH62}.
The result is known as Bogoliubov's metric \cite{PT93}.

Geodesics of the e-connection are affine in the parameters $\theta$.
If the affine combination $(1-t)\rho_\theta+t\rho_\zeta$ lies in the manifold
then it is a geodesic for the dual connection, which is called the m-connection.
IN the classical context
Chentsov \cite {CNN82} gave a characterization of the Fisher metric 
as the unique metric which is invariant
under Markov type transformations \cite {FA22}. 
On the other hand, quantum measurements are modeled by completely-positive trace-preserving maps 
acting on the manifold of density matrices. Petz \cite{PD86,PS96} gave a complete characterization
of the class of metrics which are monotone w.r.t.~these maps.
Grasselli and Streater \cite{GS01} then showed that the Bogoliubov metric is the unique element of this class
with the property that the e- and m-connections are each other dual w.r.t.~this metric.

The parameter-free approach to Information Geometry was initiated by
Pistone and Sempi \cite{PS95}. A non-commutative generalization
is studied for instance in \cite{JA01,SRF04a,SRF04b,JA06}.
These papers use the $C^*$-algebraic formulation of quantum mechanics because it
clarifies the link between classical (i.e.~non-quantum) and quantum statistics.

Areas of further research include the following.

The definition of the quantum exponential family, as given above, is not the only possibility.
It is argued in \cite{NJ22,NJ23} that the definition is highly non-unique
because of the non-uniqueness \cite{AH74} of the Radon–Nikodym derivative in a non-commutative context. 
The latter result relies on the theory of the modular operator also known as Tomita-Takesaki Theory \cite{TM70,BR87}.

Technical difficulties show up for families of density operators on an infinite-dimensional Hilbert space.
The action of the group of invertible elements of a $C^*$-algebra $\mathfrak A$ on the set
of states of $\mathfrak A$ 
induces a partition into a disjoint union of orbits each of which is a Banach manifold \cite {CIJM19}.
Exponential arcs in a manifold of quantum states are studied in \cite{NJ22,NJ23}.
These exponential arcs are candidates for being geodesics of the quantum statistical manifold.

Almost unexplored up to now is the possible impact of Quantum Information Geometry on
some of the specific models well-known in Quantum Statistical Physics.
An example in this direction is the study of scalar curvature in the transverse Ising chain \cite {NT23}.


\begin{thebibliography}{99}


\bibitem{UH62}
H. Umegaki, {\em
Conditional Expectation in an Operator Algebra.  {IV}. {E}ntropy and Information,}
Kodai Math. Sem. Rep. {\bf 14}, 59--85 (1962).

\bibitem{TM70}
M. Takesaki, {\em 
Tomita's theory of modular Hilbert algebras and its applications,}
Lecture Notes Math. {\bf 128} (Springer, 1970).

\bibitem{AH74}
H. Araki, {\em
Some properties of modular conjugation operator of von Neumann
algebras and a non-commutative Radon-Nikodym theorem with a chain rule,}
Pac. J.  Math. {\bf 50}, 309--354 (1974).

\bibitem {CNN82}
N. N. {\v C}encov, {\em
Statistical decision rules and optimal inference,}
Transl. math. monographs {\bf 53} (AMS, Providence, 1982)

\bibitem{ES85} S. Eguchi, {\em 
A differential geometric approach to statistical inference on the basis of contrast functionals,}
Hiroshima Math. J. {\bf 15}, 341--391 (1985). 

\bibitem{PD86}
D. Petz, {\em 
Quasi-entropies for Finite Quantum Systems,}
Rep. Math. Phys. {\bf 23}, 57--65 (1986)

\bibitem{BR87}
O. Bratteli and D. W. Robinson, {\em 
Operator Algebras and Quantum Statistical Mechanics 1, Second Edition,}
(Springer-Verlag, 1987)

\bibitem{PT93}
D. Petz, G. Toth, {\em
The Bogoliubov inner product in quantum statistics,}
Lett. Math. Phys. {\bf 27}, 205--216 (1993).

\bibitem {PS95}
G. Pistone, C. Sempi, {\em
An infinite-dimensional structure on the space of all 
the probability measures equivalent to a given one,}
Ann. Stat. {\bf 23}, 1543--1561 (1995).

\bibitem {PS96}
D. Petz and C.Sudar, {\em 
Geometries of Quantum States,}
J. Math. Phys. {\bf 37}, 2662--2673 (1996).

\bibitem{AN00}
S. Amari, H. Nagaoka, {\em
Methods of Information Geometry}
(Oxford University Press, 2000)
(Originally published in Japanese by Iwanami Shoten, Tokyo, Japan, 1993) 

\bibitem{GS01}
M. R. Grasselli and R. F. Streater, {\em 
On the uniqueness of the Chentsov metric in quantum information geometry,}
Infin. Dim. Anal. Quantum Prob. Rel. Top. {\bf 4}, 173--182 (2001).

\bibitem{JA01}
A. {Jen\v cov\'a}, {\em 
Geometry of quantum states: Dual connections and divergence functions,}
Rep. Math. Phys. {\bf 47}, 121--138 (2001).

\bibitem{SRF04a}
R. F. Streater, {\em  
Duality in Quantum Information Geometry,}
Open Syst. {\&} Inf. Dyn. {\bf 11}, 71--77 (2004).

\bibitem{SRF04b}	
R. F. Streater, {\em  
Quantum Orlicz Spaces in Information Geometry,}
Open Syst. {\&} Inf. Dyn. {\bf 2004}, {\bf 11}, 359--375 (2004).


\bibitem{JA06}
A. {Jen\v cov\'a}, {\em
A construction of a nonparametric quantum information manifold,}
J. Funct. Anal.  {\bf  239}, 1--20 (2006).

\bibitem{CIJM19}
F. M. Ciaglia, A. Ibort, J. Jost and G. Marmo,
{\em 
Manifolds of classical probability distributions and
quantum density operators in infinite dimensions,}
Inf. Geo. {\bf 2}, 231--271 (2019).

\bibitem{FA22}
A. Fujiwara, {\em
Hommage to Chentsov's theorem,}
Inf. Geom. {\bf 7}, Suppl. 1, 579--598 (2024).

\bibitem{NJ22}
J. Naudts, {\em
Exponential arcs in the manifold of vector states on a $\sigma$-finite von {N}eumann algebra,}
Inf. Geom. {\bf 5}, 1--30 (2022).

\bibitem{NJ23}
 J. Naudts, {\em
 Exponential arcs in manifolds of quantum states,}
Front. Phys. {\bf 11}, 1042257, (2023).

\bibitem{NT23} T. Nakamura, {\em
Monotonicity of the scalar curvature of the quantum
exponential family for transverse-field Ising chains,}
in: Geometric Science of Information, F. Nielsen and F. Barbaresco (Eds.) 
LNCS 14072 (Springer, 2023),  p. II--353.

\end{thebibliography}

\end{document}
