\documentclass[11pt]{article}
\usepackage{fullpage,url,amssymb}
\usepackage{amsmath,listings}

\usepackage{blkarray}
 
 \def\supleft#1{\ ^{#1}}
\def\bbR{\mathbb{R}}
\def\calX{\mathcal{X}}


 

\def\dq{\mathrm{d}q}
\def\dy{\mathrm{d}y}
\def\dx{\mathrm{d}x}
\def\dmu{\mathrm{d}\mu}
\def\dnu{\mathrm{d}\nu}
\def\KL{\mathrm{KL}}
\def\leftsup#1{{}^{#1}}
\def\calX{\mathcal{X}}
\def\bbR{\mathbb{R}}
\def\bbC{\mathbb{C}}
\def\bbD{\mathbb{D}}
\def\arctanh{\mathrm{arctanh}}
\def\mattwotwo#1#2#3#4{\left[\begin{array}{ll}#1 & #2 \cr #3 & #4\end{array}\right]}
\def\st{\ : \ }
\def\Conv{\mathrm{Conv}}
\def\Cone{\mathrm{Conv}}
\def\Extr{\mathrm{Extr}}
\def\Dir{\mathrm{Dir}}
\def\tr{\mathrm{tr}}
\def\barZ{\bar{Z}}
\def\vec{\mathrm{vec}}
\def\Cov{\mathhrm{Cov}}
\def\inner#1#2{\langle #1,#2\rangle}
\def\diag{\mathrm{diag}}
\def\Bhat{\mathrm{Bhat}}

\def\vectwo#1#2{\left[\begin{array}{c}#1\cr #2\end{array}\right]}
\def\JS{\mathrm{JS}}
\def\calX{\mathcal{X}}
\def\calN{\mathcal{N}}
\def\calC{\mathcal{C}}

\def\calP{\mathcal{P}}
\def\calQ{\mathcal{Q}}


\def\calF{\mathcal{F}}
\def\Var{\mathrm{Var}}
\def\calL{\mathcal{L}}
\def\dz{\mathrm{d}z}
\def\Im{\mathrm{Im}}
\def\Re{\mathrm{Re}}
\def\FR{\mathrm{FR}}
\def\SL{\mathrm{SL}}
\def\vol{\mathrm{vol}}
\def\bbB{\mathbb{B}}
\def\ds{\mathrm{d}s}

\def\deta{\mathrm{d}\eta}
\def\dtheta{\mathrm{d}\theta}

\def\Cramer{{\mbox{Cram\'er}}}

\def\Szekely{{\mbox{Sz\'ekely}}}

\def\iid{\mathrm{iid}}

\def\Hyv{\mathrm{Hyv}}

\def\PEF{\mathrm{PEF}}
\def\Nor{\mathrm{Nor}}

\def\arccosh{\mathrm{arccosh}}
\def\cosh{\mathrm{cosh}}

\lstset{
  basicstyle=\small,
  breaklines=true
}

\title{A closed-form expression of geodesics in the Klein model of hyperbolic geometry}

\author{Frank Nielsen\\ Sony Computer Science Laboratories Inc.\\ Tokyo, Japan} 
\date{April 2021}

\begin{document}
\maketitle

Hyperbolic geometry~\cite{cannon1997hyperbolic} is more and more often used in machine learning and computer vision, specially to embed and process hierarchical structures (e.g.,~\cite{suris2021hyperfuture,shimizu2020hyperbolic}).
The five main models of hyperbolic geometry~\cite{cannon1997hyperbolic} are the Poincar\'e upper space model, the Poincar\'e ball model, the Beltrami hemisphere model, the Lorentz hyperboloid model, and the Klein ball model.
These models yield metric spaces $(\bbD,d)$ where $\bbD$ denotes the domain of the model and $d(\cdot,\cdot)$ denotes the  hyperbolic distance, a metric distance.
These metric spaces are said geodesic because there exists a map $\gamma(p,q;\alpha)$ such that
$$
d\left(\gamma(p,q;s),\gamma(p,q;t)\right) = |s-t|\ d(p,q),\quad \forall p,q\in\bbD,\quad \forall s,t\in [0,1].
$$

A closed-form expression of the geodesics in the hyperbolic Poincar\'e ball model can be expressed
using M\"obius operations in the M\"obius ball gyrovector space~\cite{PoincareGeodesics-2001,shimizu2020hyperbolic}.

In this note, we report a closed-form expression of the geodesics in the Klein ball model of arbitrary dimension.
Although the Klein ball model (K) is not conformal (except at the origin), the trace of  geodesics 
$\Gamma_K(p,q)=\{(1-\alpha)p+\alpha q\ :\ \alpha\in [0,1]\}$ (called pregeodesics)
are straight line segments making it convenient for robust geometric computing (e.g., Klein hyperbolic Voronoi diagram~\cite{HVD-2010,nielsen2012hyperbolic}).
Once a structure is computed in the Klein model, it can be converted into the other models (e.g.,~\cite{HVD-2014}).

Let $\bbB_n=\{ x\in\bbR^n\ :\  x^\top x<1\}$ be the $n$-dimensional unit open ball centered at the origin.
The Klein distance $d_K(p,q)$ between point $p$ and $q$ in $\bbB_n$ (hyperbolic geometry with curvature $\kappa=-1$) is
$$
d_K(p,q)=  \arccosh \left( \frac{1-p^\top q}{\sqrt{(1-p^\top p)}\sqrt{(1-q^\top q)}} \right).
$$

Thus we seek a parameterization 
\begin{equation}
\gamma_K(p,q;\alpha)= (1-u(\alpha)) p + u(\alpha)q
\end{equation}
so that 
$$
d_K\left(\gamma_K(p,q;s),\gamma_K(p,q;t)\right) = |s-t|\ d_K(p,q).
$$

In particular, when $s=0$ and $t=\alpha$, we shall have
$$
d_K\left(p,(1-u(\alpha)) p + u(\alpha)q\right)=\alpha d_K(p,q).
$$

This latter equation amounts to solve for $u(\alpha)$ in the equation:
$$
\frac{a-b u(\alpha)}{\sqrt{a(a-2b u(\alpha)+c u(\alpha)^2)}}-d(\alpha)=0,
$$
where
\begin{eqnarray*}
a &=& 1-p^\top p,\\
b &=& p^\top (q-p),\\
c &=& (q-p)^\top (q-p) ,\\
d(\alpha) &=& \cosh(\alpha d_K(p,q))
\end{eqnarray*}

Using symbolic calculations, we find the following solution:
\begin{equation}
u(\alpha) = \frac{   ad(\alpha) \sqrt{(ac+b^2)  (d(\alpha)^2-1)} +a b (1-d(\alpha)^2)}{a c d(\alpha)^2 + b^2}.
\end{equation}

Thus we get in closed-form the Klein geodesics $\gamma_K$ (albeit a large formula) such that
$$
d_K(\gamma(p,q;s),\gamma(p,q;t))= |s-t|\ d_K(p,q).
$$

The snippet code below implements in {\sc Maxima}\footnote{\url{https://maxima.sourceforge.io/}} the  geodesics in the Klein model with a test set.

\begin{lstlisting}
dKlein(p,q):=acosh((1-p.q)/(sqrt((1-p.p)*(1-q.q))));

u(p,q,alpha):= ((1-p.p)*cosh(alpha* dKlein(p,q))*sqrt(((1-p.p)*((q-p).(q-p))+(p.(q-p))**2)*  (cosh(alpha* dKlein(p,q))
**2-1)) +(1-p.p)* (p.(q-p))* (1-cosh(alpha* dKlein(p,q))**2))/((1-p.p)* ((q-p).(q-p)) * cosh(alpha* dKlein(p,q))**2 + (p.(q-p))**2);

gammaKlein(p,q,alpha):=(1-u(p,q,alpha))*p+u(p,q,alpha)*q;

/* Test */
p: [0.5, 0.2];
q: [0.1, -0.3];

/* 1st test for Klein geodesics */
alpha:random(1.0);
alpha*dKlein(p,q);
dKlein(p,gammaKlein(p,q,alpha));

/* 2nd test for Klein geodesics */
s:random(1.0);
t:random(1.0);
dKlein(gammaKlein(p,q,s),gammaKlein(p,q,t));
abs(s-t)*dKlein(p,q);
\end{lstlisting} 

To illustrate the use of these Klein geodesics, consider computing the smallest enclosing ball of a finite point set 
$\calP=\{p_1,\ldots,p_m\}$ in hyperbolic geometry~\cite{ArnaudonNielsen-2012}. 
The closed-form expression of the Klein geodesic, allows to bypass the use of hyperbolic translations from/to the ball origin as this was used  in~\cite{NielsenHadjeres-2015}.
The algorithm for calculating an approximation of the hyperbolic smallest enclosing ball is:
\begin{itemize}
\item Initialize $c_1=p_1$
\item Repeat $t$ times: Let $c_{i+1}=\gamma_K\left(c_i,p_{f_i},\frac{1}{i+1}\right)$ where $p_{f_i}$ is the farthest point of $\calP$ to $c_i$.
That is, we have $f_i=\arg\max_{j\in\{1,\ldots, m\}} d_K(c_i,p_j)$.
\end{itemize}

The algorithm is proven to converge in~\cite{ArnaudonNielsen-2012} (i.e., $\lim_{t\rightarrow\infty} c_t=\arg \min_c \max_{i\in\{1,\ldots,m\}} d_K(p_i,c)$) since the hyperbolic geometry is a Hadamard space. 

\vskip 0.5cm
\noindent Additional material is available online at \url{https://franknielsen.github.io/KleinGeodesics/}

\bibliographystyle{plain}
\bibliography{ClosedFormKleinGeodesicsBIB}


\end{document}



 
