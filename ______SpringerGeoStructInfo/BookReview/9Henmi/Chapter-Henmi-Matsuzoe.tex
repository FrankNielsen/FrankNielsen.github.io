%%%%%%%%%%%%%%%%%%%% author.tex %%%%%%%%%%%%%%%%%%%%%%%%%%%%%%%%%%%%%%%
%
% sample root file for your "contribution" to a contributed volume
%
% Use this file as a template for your own input.
%
%%%%%%%%%%%%%%%%%%%% Springer %%%%%%%%%%%%%%%%%%%%%%%%%%%%%%%%%%%%%%%%%


% RECOMMENDED %%%%%%%%%%%%%%%%%%%%%%%%%%%%%%%%%%%%%%%%%%%%%%%%%%%%%
\documentclass[graybox]{svmult}

% choose options for [] as required from the list
% in the Reference Guide

\usepackage{mathptmx}       % selects Times Roman as basic font
\usepackage{helvet}         % selects Helvetica as sans-serif font
\usepackage{courier}        % selects Courier as typewriter font
\usepackage{type1cm}        % activate if the above 3 fonts are
                            % not available on your system
%
\usepackage{makeidx}         % allows index generation
\usepackage{graphicx}        % standard LaTeX graphics tool
                             % when including figure files
\usepackage{multicol}        % used for the two-column index
\usepackage[bottom]{footmisc}% places footnotes at page bottom

\newcommand{\bm}[1]{\mbox{\boldmath $#1$}}
\newcommand{\bms}[1]{\mbox{\scriptsize\boldmath $#1$}}

% see the list of further useful packages
% in the Reference Guide

\makeindex             % used for the subject index
                       % please use the style svind.ist with
                       % your makeindex program

%%%%%%%%%%%%%%%%%%%%%%%%%%%%%%%%%%%%%%%%%%%%%%%%%%%%%%%%%%%%%%%%%%%%%%%%%%%%%%%%%%%%%%%%%

\begin{document}

\title*{Statistical Manifolds Admitting Torsion and Partially Flat Spaces}
\author{Masayuki Henmi and Hiroshi Matsuzoe}
\institute{Masayuki Henmi \at The Institute of Statistical Mathematics,
10-3 Midori-cho, Tachikawa, Tokyo 190-8562, Japan \\
\email{henmi@ism.ac.jp}
\and Hiroshi Matsuzoe \at Nagoya Institute of Technology,
gokiso-cho, showa-ku, Nagoya, Aichi 466-8555, Japan \\
\email{matsuzoe@nitech.ac.jp}}
\maketitle

\abstract{It is well-known that a contrast function defined on a product manifold 
$M \times M$ induces a Riemannian metric and a pair of dual torsion-free affine 
connections on the manifold $M$. This geometrical structure is called a statistical 
manifold and plays a central role in information geometry.
Recently, the notion of pre-contrast function has been introduced and shown to
induce a similar differential geometrical structure on $M$, but one of the two dual
affine connections is not necessarily torsion-free. This structure is called a statistical 
manifold admitting torsion. The notion of statistical manifolds admitting torsion has
been originally introduced to study a geometrical structure which appears in a quantum
statistical model. However, it has been shown that an estimating function which is
used in ``classical'' statistics also induces a statistical manifolds admitting torsion
through its associated pre-contrast function.
The aim of this paper is to summarize such previous results. In particular, we focus
on a partially flat space, which is a statistical manifolds admitting torsion where one
of its dual connections is flat. In this space, it is possible to discuss some properties     
similar to those in a dually flat space, such as canonical a pre-contrast function and
a generalized projection theorem.}

%%%%%%%%%%%%%%%%%%%%%%%%%%%%%%%%%%%%%%%%%%%%%%%%%%%%%%%%%%%%%%%%%%%%%%%%%%%%%%%%%%%%%%%%%%%
%%%%%%%% Introduction %%%%%%%%%%%%%%%%%%%%%%%%%%%%%%%%%%%%%%%%%%%%%%%%%%%%%%%%%%%%%%%%%%%%%%%%
%%%%%%%%%%%%%%%%%%%%%%%%%%%%%%%%%%%%%%%%%%%%%%%%%%%%%%%%%%%%%%%%%%%%%%%%%%%%%%%%%%%%%%%%%%%
\section{Introduction}
\label{sec:1}
A statistical manifold is a Riemannian manifold with a pair of two dual torsion-free affine
connections and it plays a central role in information geometry.
This geometrical structure is induced from an asymmetric (squared) distance-like smooth
function called a contrast function by taking its second and third derivatives \cite{E},
\cite{Ma1}.
The Kullback-Leibler divergence on a regular parametric statistical model is a typical
example of contrast functions and its induced geometrical objects are the Fisher metric,
the exponential and mixture connections. The geometrical structure determined by these
objects plays an important role in the geometry of statistical inference, as is widely
known \cite{AN}, \cite{A}.

A statistical manifold admitting torsion (SMAT) is a Riemannian manifold with a pair of
two dual affine connections, where only one of them must be torsion-free but the other
is necessarily {\it not}. This geometrical structure naturally appears in a quantum
statistical model (i.e. a set of density matrices representing quantum states) \cite{AN}
and the notion of SMAT was originally introduced to study such a geometrical structure 
from a mathematical point of view \cite{Ku2}. A pre-contrast function was subsequently
introduced as a generalization for the first derivative of a contrast function and it was
shown that an pre-contrast function induces a SMAT by taking its first and second
derivatives \cite{Ma2}.

In statistics, an estimating function is a function defined on a direct product of
parameter and sample spaces, and it is used to obtain an estimator by solving its 
corresponding estimating equation.
Henmi and Matsuzoe \cite{HM} showed that a SMAT also appears in ``classical"
statistics through an estimating function. More precisely, an estimating function
naturally defines a pre-contrast function on a parametric statistical model and a
SMAT is induced from it.

This paper summarizes such previous results, focusing on a SMAT where one
of its dual connections is flat. We call this geometrical structure a partially flat space.
Although this space is different from a dually flat space in general since one of the
dual connections in a SMAT has possibly torsion, some similar properties hold.
For example, the canonical pre-contrast function can be naturally defined on a
partially flat space, which is an analog of the canonical contrast function (or canonical
divergence) in a dually flat space. In addition, a generalized projection theorem holds
with respect to the canonical pre-contrast function. This theorem can be seen as a
generalization of the projection theorem in a dually flat space.
This paper is an extended version of the conference proceeding \cite{He}.
We consider a statistical problem to see an example of statistical manifolds admitting
torsion induced from estimating functions and discuss some future problems, both of
which were not included in \cite{He}.

%%%%%%%%%%%%%%%%%%%%%%%%%%%%%%%%%%%%%%%%%%%%%%%%%%%%%%%%%%%%%%%%%%%%%%%%%%%%%%%%%%%%%%%%%%%
%%%%%%%% Statistical Manifolds and Contrast Functions %%%%%%%%%%%%%%%%%%%%%%%%%%%%%%%%%%%%%%%%%%%%%
%%%%%%%%%%%%%%%%%%%%%%%%%%%%%%%%%%%%%%%%%%%%%%%%%%%%%%%%%%%%%%%%%%%%%%%%%%%%%%%%%%%%%%%%%%%
\section{Statistical Manifolds and Contrast Functions}
\label{sec:2}
Through this paper, we assume that all geometrical objects on differentiable manifolds are
smooth and restrict our attention to Riemannian manifolds, although the most of the concepts
can be defined for semi-Riemannian manifolds.

Let $(M,g)$ be a Riemannian manifold and $\nabla$ be an affine connection on $M$. The {\em
dual connection} $\nabla^{\ast}$ of $\nabla$ with respect to $g$ is defined by
%
\begin{eqnarray*}
   Xg(Y,Z) = g(\nabla_{X}Y, Z) + g(Y, \nabla^{\ast}_{X}Z) \ \ \
   (\forall X,\forall Y,\forall Z \in {\cal X}(M))
\end{eqnarray*}
%
where ${\cal X}(M)$ is the set of all vector fields on $M$.

For a affine connection $\nabla$ on $M$, its curvature tensor field $R$ and torsion tensor
field $T$ are defined by the following equations as usual:
%
\begin{eqnarray*}
   R(X,Y)Z &:=& \nabla_{X}\nabla_{Y}Z - \nabla_{Y}\nabla_{X}Z - \nabla_{[X,Y]}Z, \\
   T(X,Y) &:=& \nabla_{X}Y - \nabla_{Y}X - [X,Y]
\end{eqnarray*}
%
($\forall X,\forall Y,\forall Z \in {\cal X}(M)$).
It is said that an affine connection $\nabla$ is {\em torsion-free} if $T=0$.
Note that for a torsion-free affine connection $\nabla$, $\nabla^{\ast}=\nabla$ implies
that $\nabla$ is the Levi-Civita connection with respect to $g$.
Let $R^{\ast}$ and $T^{\ast}$ be the curvature and torsion tensor fields of
$\nabla^{\ast}$, respectively. It is easy to see that $R=0$ always implies $R^{\ast}=0$,
but $T=0$ does not necessarily implies $T^{\ast}=0$.

Let $\nabla$ be a torsion-free affine connection on a Riemannian manifold $(M,g)$.
Following \cite{Ku1},
we say that $(M,g,\nabla)$ is a {\em statistical manifold} if and only if $\nabla g$ is a
symmetric $(0,3)$-tensor field, that is
%
\begin{eqnarray}\label{SM}
   (\nabla_{X}g)(Y,Z) = (\nabla_{Y}g)(X,Z) \ \ \ (\forall X,\forall Y,\forall Z \in {\cal X}(M)).
\end{eqnarray}
%
This condition is equivalent to $T^{\ast}=0$ under the condition that $\nabla$ is a torsion-free.
If $(M,g,\nabla)$ is a statistical manifold, so is $(M,g,\nabla^{\ast})$ and it is called the
{\em dual statistical manifold} of $(M,g,\nabla)$. Since $\nabla$ and $\nabla^{\ast}$ are both
torsion-free for a statistical manifold $(M,g,\nabla)$, $R=0$ implies that
$\nabla$ and $\nabla^{\ast}$ are both flat. In this case, $(M,g,\nabla,\nabla^{\ast})$ is called
a {\em dually flat space} \cite{AN}.

Let $\phi$ be a real-valued function on the direct product $M \times M$ of a manifold $M$
and $X_1,...,X_i,Y_1,...,Y_j$ be vector fields on $M$. The functions
$\phi[X_1,...,X_i|Y_1,...,Y_j]$, $\phi[X_1,...,X_i| \ ]$ and $\phi[ \ |Y_1,...,Y_j]$ on
$M$ are defined by the equations
%
\begin{eqnarray}
   \phi[X_1,\ldots,X_i | Y_1,\ldots,Y_j](r) &:=&
   (X_1)_p \cdots (X_i)_p(Y_1)_q \cdots (Y_j)_q\phi(p,q)|_{p=r,q=r}, \label{Eguchi1} \\
   \phi[X_1,\ldots,X_i | \ ](r) &:=&
   (X_1)_p \cdots (X_i)_p\phi(p,r)|_{p=r}, \label{Eguchi2} \\
   \phi[ \ |Y_1,\ldots,Y_j](r) &:=&
   (Y_1)_q \cdots (Y_j)_q\phi(r,q)|_{q=r} \label{Eguchi3}
\end{eqnarray}
%
for any $r \in M$, respectively \cite{E}.
Using these notations, a {\em contrast function} $\phi$ on $M$ is defined to be a real-valued
function on $M \times M$ which satisfies the following conditions \cite{E}, \cite{Ma1}:
%
\begin{eqnarray*}
   &(a)& \phi(p,p) = 0 \ \ \ (\forall p \in M), \\
   &(b)& \phi[X| \ ] = \phi[ \ |X] = 0 \ \ \ (\forall X \in {\cal X}(M)), \\
   &(c)& \mbox{$g(X,Y) := -\phi[X|Y] \ \ (\forall X, \forall Y \in {\cal X}(M))$
               is a Riemannian metric on $M$}.
\end{eqnarray*}
%
Note that these conditions imply that
%
\begin{eqnarray*}
   \phi(p,q) \ge 0, \ \ \phi(p,q) = 0 \Longleftrightarrow p=q
\end{eqnarray*}
%
in some neighborhood of the diagonal set $\{(r,r) | r \in M\}$ in $M \times M$.
Although a contrast function is not necessarily symmetric, this property means that
a contrast function measures some discrepancy between two points on $M$ (at least locally).
For a given contrast function $\phi$, the two affine connections $\nabla$ and $\nabla^{\ast}$ 
are defined by
%
\begin{eqnarray*}
   g(\nabla_{X}Y,Z) = -\phi[XY|Z], \ \
   g(Y,\nabla^{\ast}_{X}Z) = -\phi[Y|XZ]
\end{eqnarray*}
%
($\forall X, \forall Y, \forall Z \in {\cal X}(M)$).
In this case, $\nabla$ and $\nabla^{\ast}$ are both torsion-free and dual to each other with
respect to $g$. This means that both of $(M,g,\nabla)$ and $(M,g,\nabla^{\ast})$ are statistical
manifolds. In particular, $(M,g,\nabla)$ is called the statistical manifold induced from the
contrast function $\phi$.

A typical example of contrast functions is the Kullback-Leibler divergence on a statistical model.
Let
$S=\{p(\bm{x};\bm{\theta}) \ | \ \bm{\theta}=(\theta^1,...,\theta^d) \in \Theta \subset \bm{R}^d\}$
be a regular parametric statistical model, which is a set of probability density functions with
respect to a dominating measure $\nu$ on a sample space $\Omega$. Each element is indexed by a
parameter (vector) $\bm{\theta}$ in an open subset $\Theta$ of $\bm{R}^d$ and the set $S$ satisfies
some regularity conditions, under which $S$ can be seen as a differentiable manifold. 
The Kullback-Leibler divergence of the two density functions $p_1(\bm{x})=p(\bm{x};\bm{\theta}_1)$
and $p_2(\bm{x})=p(\bm{x};\bm{\theta}_2)$ in $S$ is defined to be
%
\begin{eqnarray*}\label{KL}
\phi_{KL}(p_1,p_2) := \int_{\Omega}p_2(\bm{x})\log\frac{p_2(\bm{x})}{p_1(\bm{x})}\nu(d\bm{x}).
\end{eqnarray*}
%
It is easy to see that the Kullback-Leibler divergence satisfies the conditions $(a)$, $(b)$
and $(c)$, and so it is a contrast function on $S$. Its induced Riemannian metric and dual
connectins are Fisher metric $g^F$, the exponential connection $\nabla^{(e)}$ and mixture
connection $\nabla^{(m)}$, respectively. They are given as follows:
%
\begin{eqnarray*}
   &&g_{jk}^F(\bm{\theta}) := g^F(\partial_j,\partial_k)
                            = E_{\bms{\theta}}\{s^j(\bm{x},\bm{\theta})s^k(\bm{x},\bm{\theta})\},
                            \label{Fisher} \\
   &&\left\{\begin{array}{l}
               \Gamma_{ij,k}^{(e)}(\bm{\theta}) 
               := g^F(\nabla_{\partial_i}^{(e)}\partial_j,\partial_k)
                = E_{\bms{\theta}}[\{\partial_is^j(\bm{x},\bm{\theta})\}s^k(\bm{x},\bm{\theta})] \\
               \Gamma_{ik,j}^{(m)}(\bm{\theta})
               := g^F(\partial_j,\nabla_{\partial_i}^{(m)}\partial_k)
                = \int_{\Omega}s^j(\bm{x},\bm{\theta})\partial_i\partial_kp(\bm{x};\bm{\theta})
                  \nu(d\bm{x})
            \end{array}
     \right., \label{emconn}
\end{eqnarray*}
%
where $E_{\bms{\theta}}$ indicates that the expectation is taken with respect to
$p(\bm{x};\bm{\theta})$, $\partial_i=\frac{\partial}{\partial\theta^i}$ and
$s^{i}(\bm{x};\bm{\theta})=\partial_{i}\log p(\bm{x};\bm{\theta}) \ (i=1,\ldots,d)$.
As is widely known, this geometrical structure plays the most fundamental and important role
in the differential geometry of statistical inference \cite{AN}, \cite{A}.

%%%%%%%%%%%%%%%%%%%%%%%%%%%%%%%%%%%%%%%%%%%%%%%%%%%%%%%%%%%%%%%%%%%%%%%%%%%%%%%%%%%%%%%%%%%
%%%%%%%% Statistical Manifolds Admitting Torsion and Pre-contrast Functions %%%%%%%%%%%%%%%%%%%%%%%%%%
%%%%%%%%%%%%%%%%%%%%%%%%%%%%%%%%%%%%%%%%%%%%%%%%%%%%%%%%%%%%%%%%%%%%%%%%%%%%%%%%%%%%%%%%%%%
\section{Statistical Manifolds Admitting Torsion and Pre-contrast Functions}
\label{sec:3}
A statistical manifold admitting torsion is an abstract notion for the geometrical structure
where only one of the dual connections is allow to have torsion, which naturally appears in
a quantum statistical model \cite{AN}. The definition is obtained by generalizing (\ref{SM})
in the definition of statistical manifold as follows \cite{Ku2}.

Let $(M,g)$ be a Riemannian manifold and $\nabla$ be an affine connection on $M$.
We say that $(M,g,\nabla)$ is a {\em statistical manifold admitting torsion} (SMAT for short)
if and only if
%
\begin{eqnarray}\label{SMAT}
   (\nabla_{X}g)(Y,Z) - (\nabla_{Y}g)(X,Z) = -g(T(X,Y),Z) \ \ \
   (\forall X,\forall Y,\forall Z \in {\cal X}(M)).
\end{eqnarray}
%
This condition is equivalent to $T^{\ast}=0$ in the case where $\nabla$ possibly has torsion,
and it reduces to (\ref{SM}) if $\nabla$ is torsion-free.
Note that $(M,g,\nabla^{\ast})$ is not necessarily a statistical manifold although $\nabla^{\ast}$
is torsion-free. It should be also noted that $(M,g,\nabla^{\ast})$ is a SMAT whenever a
torsion-free affine connection $\nabla$ is given on a Riemannian manifold $(M,g)$.

For a SMAT $(M,g,\nabla)$, $R=0$ does not necessarily imply that $\nabla$ is flat, but it
implies that $\nabla^{\ast}$ is flat since $R^{\ast}=0$ and $T^{\ast}=0$. In this case,
we call $(M,g,\nabla,\nabla^{\ast})$ a {\em partially flat space}.

Let $\rho$ be a real-valued function on the direct product $TM \times M$ of a manifold
$M$ and its tangent bundle $TM$, and $X_1,...,X_i,Y_1,...,Y_j,Z$ be vector fields on $M$.
The function $\rho[X_1,...,X_iZ|Y_1,...,Y_j]$ on $M$ is defined by
%
\begin{eqnarray*}
   \rho[X_1,\ldots,X_iZ | Y_1,\ldots,Y_j](r) := (X_1)_p \cdots (X_i)_p(Y_1)_q \cdots (Y_j)_q
\rho(Z_p,q)|_{p=r,q=r}
\end{eqnarray*}
%
for any $r \in M$.
Note that the role of $Z$ is different from those of the vector fields in the notation of 
(\ref{Eguchi1}). The functions $\rho[X_1,...,X_iZ| \ ]$ and $\rho[ \ |Y_1,...,Y_j]$ are also
defined in the similar way to (\ref{Eguchi2}) and (\ref{Eguchi3}).

We say that $\rho$ is a {\em pre-contrast function} on $M$ if and only if the following
conditions are satisfied \cite{Ma2}, \cite{HM}:
%
\begin{eqnarray*}
   &(a)& \rho(f_1X_1+f_2X_2,q) = 0 \ \
         (\forall f_1, \forall f_2 \in C^{\infty}(M), \ \forall X_1, \forall X_2 \in {\cal X}(M), \ \forall q \in M). \\
   &(b)& \rho[X| \ ] = 0 \ \ (\forall X \in {\cal X}(M)) \ \ \
         \left({\it i.e.} \ \rho(X_p,p)=0 \ \ (\forall p \in M)\right). \\
   &(c)& \mbox{$g(X,Y) := -\rho[X|Y] \ \ (\forall X, \forall Y \in {\cal X}(M))$
               is a Riemannian metric on $M$}.
\end{eqnarray*}
%
Note that for any contrast function $\phi$ on $M$, the function $\rho_{\phi}$ which is defined by 
%
\begin{eqnarray*}
   \rho_{\phi}(X_p,q) := X_p\phi(\cdot,q) \ \ \ (\forall p, \forall q \in M, \ \forall X_p \in T_p(M))
\end{eqnarray*}
%
is a pre-contrast function on $M$.
The notion of pre-contrast function is obtained by taking the fundamental properties of the first
derivative of a contrast function as axioms.
For a given pre-contrast function $\rho$, two affine connections $\nabla$ and $\nabla^{\ast}$ are
defined by
%
\begin{eqnarray*}
   g(\nabla_{X}Y,Z) = -\rho[XY|Z], \ \ g(Y,\nabla^{\ast}_{X}Z) = -\rho[Y|XZ]
\end{eqnarray*}
%
($\forall X, \forall Y, \forall Z \in {\cal X}(M)$) in the same way as a contrast function.
In this case, $\nabla$ and $\nabla^{\ast}$ are dual to each other with respect to $g$ and
$\nabla^{\ast}$ is torsion-free. However, the affine connection $\nabla$ possibly has
torsion. This means that $(M,g,\nabla)$ is a SMAT and it is called the SMAT induced from
the pre-contrast function $\rho$.

%%%%%%%%%%%%%%%%%%%%%%%%%%%%%%%%%%%%%%%%%%%%%%%%%%%%%%%%%%%%%%%%%%%%%%%%%%%%%%%%%%%%%%%%%
%%%%%%%% Canonical Pre-contrast functions in Partially Flat Spaces %%%%%%%%%%%%%%%%%%%%%%%%%%%%%%%%%
%%%%%%%%%%%%%%%%%%%%%%%%%%%%%%%%%%%%%%%%%%%%%%%%%%%%%%%%%%%%%%%%%%%%%%%%%%%%%%%%%%%%%%%%%
\section{Canonical Pre-contrast Functions in Partially Flat Spaces}
\label{sec:4}
In a dually flat space $(M,g,\nabla,\nabla^{\ast})$, it is well-known that the canonical
contrast functions (called $\nabla$ and $\nabla^{\ast}$- divergences) are naturally
defined, and the Pythagorean theorem and the projection theorem are stated in terms of
the $\nabla$ and $\nabla^{\ast}$- geodesics and the canonical contrast functions \cite{AN},
\cite{A}.
In a partially flat space $(M,g,\nabla,\nabla^{\ast})$, where $R=R^{\ast}=0$ and
$T^{\ast}=0$, it is possible to define a pre-contrast function which can be seen as
canonical, and a projection theorem holds with respect to the ``canonical" pre-contrast
function and the $\nabla^{\ast}$-geodesic.
%
\begin{proposition}[Canonical Pre-contrast Functions]
   Let $(M,g,\nabla,\nabla^{\ast})$ be a partially flat space (i.e. $(M,g,\nabla)$
   is a SMAT with $R=R^{\ast}=0$ and $T^{\ast}=0$) and $(U,\eta_i)$ be an affine
   coordinate neighborhood with respect to $\nabla^{\ast}$ in $M$.
   The function $\rho$ on $TU \times U$ defined by
%
\begin{eqnarray}
      \rho(Z_p, q) := - g_p(Z_p, \dot{\gamma}^{\ast}(0)) \ \ 
      (\forall p, \forall q \in U, \forall Z_p \in T_p(U)), \label{canpre}
\end{eqnarray}
%
is a pre-contrast function on $U$,
where $\gamma^{\ast}:[0,1] \rightarrow U$ is the $\nabla^{\ast}$-geodesic such that
$\gamma^{\ast}(0)=p, \gamma^{\ast}(1)=q$ and $\dot{\gamma}^{\ast}(0)$ is the tangent
vector of $\gamma^{\ast}$ on $p$.
Furthermore, the pre-contrast function $\rho$ induces the original Riemannian metric
$g$ and the dual connections $\nabla$ and $\nabla^{\ast}$ on $U$.
\end{proposition}
%
\begin{proof}%[of Proposition 1]
   For the function $\rho$ defined as (\ref{canpre}), the condition $(a)$ in the definition
   of pre-contrast functions follows from the bilinearity of the inner product $g_p$.
   The condition $(b)$ immediately follows from $\dot{\gamma}^{\ast}(0)=0$ when $p=q$.
   By calculating the derivatives of $\rho$ with the affine coordinate system $(\eta_i)$,
   it can be shown that the condition $(c)$ holds and that the induced Riemannian metric
   and dual affine connections coincide with the original $g$, $\nabla$ and $\nabla^{\ast}$.
   \qed
\end{proof}
%
In particular, if $(U,g,\nabla,\nabla^{\ast})$ is a dually flat space, the pre-contrast
function $\rho$ defined in (\ref{canpre}) coincides with the directional derivative
$Z_p\phi^{\ast}(\cdot,q)$ of $\nabla^{\ast}$-divergence $\phi^{\ast}(\cdot,q)$ with
respect to $Z_p$ (cf. \cite{HK}, \cite{AA}).
Hence, the definition of (\ref{canpre}) seems to be natural one and we call the function
$\rho$ in (\ref{canpre}) the {\em canonical pre-contrast function} in a partially flat
space $(U,g,\nabla,\nabla^{\ast})$.

From the definition of the canonical pre-contrast function, we can immediately obtain
the following theorem.
%
\begin{corollary}[Generalized Projection Theorem]
   Let $(U,\eta_i)$ be an affine coordinate neighborhood in a partially flat space 
   $(M,g,\nabla,\nabla^{\ast})$ and $\rho$ be the canonical pre-contrast function 
   on $U$. For any submanifold $N$ in $U$, the following
   conditions are equivalent:
%
\begin{eqnarray*}
   &(i)& \mbox{The $\nabla^{\ast}$-geodesic starting at $q \in U$ is perpendicular to $N$
               at $p \in N$}. \\
   &(ii)& \ \rho(Z_p, q) = 0 \ \mbox{for any $Z_p$ in $T_p(N)$.} \label{gpthm}
\end{eqnarray*}
\end{corollary}
%
If $(U,g,\nabla,\nabla^{\ast})$ is a dually flat space, this theorem reduces to the
projection theorem with respect to the $\nabla^{\ast}$-divergence $\phi^{\ast}$, since
$\rho(Z_p, q)=Z_p\phi^{\ast}(\cdot,q)$.
In this sense, it can be seen as a generalized version of the projection theorem in
dually flat spaces, and this is also one of the reasons why we consider the pre-contrast
function $\rho$ defined in (\ref{canpre}) as canonical.

%%%%%%%%%%%%%%%%%%%%%%%%%%%%%%%%%%%%%%%%%%%%%%%%%%%%%%%%%%%%%%%%%%%%%%%%%%%%%%%%%%%%%%%%%%
%%%%%%%% Statistical Manifolds Admitting Torsion Induced from Estimating Functions %%%%%%%%%%%%%%%%%%%%
%%%%%%%%%%%%%%%%%%%%%%%%%%%%%%%%%%%%%%%%%%%%%%%%%%%%%%%%%%%%%%%%%%%%%%%%%%%%%%%%%%%%%%%%%%
\section{Statistical Manifolds Admitting Torsion Induced from Estimating Functions}
\label{sec:5}
As we mentioned in Introduction, a SMAT naturally appears through an estimating function in
a ``classical" statistical model as well as in a quantum statistical model. In this section,
we briefly explain how a SMAT is induced on a parametric statistical model from an estimating
function. See \cite{HM} for more details.

Let $S=\{p(\bm{x};\bm{\theta}) \ | \ \bm{\theta}=(\theta^1,...,\theta^d) \in \Theta \subset
\bm{R}^d\}$ be a regular parametric statistical model.
An estimating function on $S$, which we consider here, is a $\bm{R}^{d}$-valued function
$\bm{u}(\bm{x},\bm{\theta})$ satisfying the following conditions:
%
\begin{eqnarray*}
   E_{\bms{\theta}}\{\bm{u}(\bm{x},\bm{\theta})\} = \bm{0}, \ \
   E_{\bms{\theta}}\{\|\bm{u}(\bm{x},\bm{\theta})\|^2\} < \infty, \ \
   \det\left[E_{\bms{\theta}}\left\{\frac{\partial\bm{u}}{\partial\bm{\theta}}
   (\bm{x},\bm{\theta})\right\}\right] \neq 0 \ \
   (\forall\bm{\theta} \in \Theta).
\end{eqnarray*}
%
The first condition is called the unbiasedness of estimating functions, which is important
to ensure the consistency of the estimator obtained from an estimating function.
Let $X_1,\ldots,X_n$ be a random sample from an unknown probability distribution
$p(\bm{x};\bm{\theta}_0)$ in $S$. The estimator $\hat{\bm{\theta}}$ for $\bm{\theta}_0$
is called an M-estimator if it is obtained as a solution to the estimating equation
%
\begin{eqnarray}\label{esteq}
   \sum_{i=1}^{n}\bm{u}(\bm{X}_i,\bm{\theta}) = \bm{0}.
\end{eqnarray}
%
The M-estimator $\hat{\bm{\theta}}$ has the consistency
%
\begin{eqnarray*}
   \hat{\bm{\theta}} \longrightarrow \bm{\theta}_0 \ \ (\mbox{in probability})
\end{eqnarray*}
%
as $n \rightarrow \infty$ and the asymptotic normality
%
\begin{eqnarray*}
   \sqrt{n}(\hat{\bm{\theta}} - \bm{\theta}_0) \longrightarrow
   N\left(\bm{0},{\rm Avar}\left(\hat{\bm{\theta}}\right)\right) \ \ (\mbox{in distribution})
\end{eqnarray*}
%
as $n \rightarrow \infty$ under some additional regularity conditions \cite{V}, where
${\rm Avar}(\hat{\bm{\theta}})$ is the asymptotic variance-covariance matrix of
$\hat{\bm{\theta}}$ and is given by
%
\begin{eqnarray}\label{avar}
   {\rm Avar}(\hat{\bm{\theta}}) = \{A(\bm{\theta}_0)\}^{-1}B(\bm{\theta}_0)
                                   \{A(\bm{\theta}_0)\}^{-T} \label{avar}
\end{eqnarray}
%
with $A(\bm{\theta}):=E_{\bms{\theta}}\left\{(\partial\bm{u}/\partial\bm{\theta})(\bm{x},\bm{\theta})\right\}$
and $B(\bm{\theta}):=E_{\bms{\theta}}\left\{\bm{u}(\bm{x},\bm{\theta})\bm{u}(\bm{x},\bm{\theta})^T\right\}$.

In order to induce the structure of SMAT on $S$ from an estimating function, we consider
the notion of {\em standardization} of estimating functions. For an estimating function
$\bm{u}(\bm{x},\bm{\theta})$, its standardization
(or {\em standardized estimating function}) is defined by
%
\begin{eqnarray*}
   \bm{u}_{\ast}(\bm{x},\bm{\theta})
   := E_{\bms{\theta}}\left\{\bm{s}(\bm{x},\bm{\theta})\bm{u}(\bm{x},\bm{\theta})^T\right\}
      \left[E_{\bms{\theta}}\left\{\bm{u}(\bm{x},\bm{\theta})\bm{u}
      (\bm{x},\bm{\theta})^T\right\}\right]^{-1}\bm{u}(\bm{x},\bm{\theta}),
\end{eqnarray*}
%
where $\bm{s}(\bm{x},\bm{\theta})=(\partial/\partial\bm{\theta})\log p(\bm{x};\bm{\theta})$
is the score function for $\bm{\theta}$ \cite{H}.
Geometrically, the $i$-th component of the standardized estimating function
$\bm{u}_{\ast}(\bm{x},\bm{\theta})$ is the orthogonal projection of the $i$-th component of
the score function $\bm{s}(\bm{x},\bm{\theta})$ onto the linear space spanned by
all components of the estimating function $\bm{u}(\bm{x},\bm{\theta})$ in the Hilbert space
%
\begin{eqnarray*}
   {\cal H}_{\bms{\theta}} := \{a(\bm{x}) \ | \ E_{\bms{\theta}}\{a(\bm{x})\} = 0, \
                                E_{\bms{\theta}}\{a(\bm{x})^2\} < \infty\}
\end{eqnarray*}
%
with the inner product
$<a(\bm{x}), b(\bm{x})>_{\bms{\theta}}:=E_{\bms{\theta}}\{a(\bm{x})b(\bm{x})\} \
(\forall a(\bm{x}), \forall b(\bm{x}) \in {\cal H}_{\bms{\theta}})$.
The standardization $\bm{u}_{\ast}(\bm{x},\bm{\theta})$ of $\bm{u}(\bm{x},\bm{\theta})$ does
not change the estimator since the estimating equation obtaind from
$\bm{u}_{\ast}(\bm{x},\bm{\theta})$ is equivalent to the original estimating equation
(\ref{esteq}).
In terms of the standardization, the asymptotic variance-covariance matrix (\ref{avar}) can
be rewritten as
%
\begin{eqnarray*}
   {\rm Avar}(\hat{\bm{\theta}}) = \{G(\bm{\theta}_0)\}^{-1},
\end{eqnarray*}
%
where $G(\bm{\theta}):=E_{\bms{\theta}}\left\{\bm{u}_{\ast}(\bm{x},\bm{\theta})\bm{u}_{\ast}
(\bm{x},\bm{\theta})^T\right\}$.
The matrix $G(\bm{\theta})$ is called a Godambe information matrix \cite{G}, which can be seen
as a generalization of the Fisher information matrix.

As we have seen in Section 2, the Kullback-Leibler divergence $\phi_{KL}$ is a contrast 
function on $S$. Hence, the first derivative of $\phi_{KL}$ is a pre-contrast function on 
$S$ and given by
%
\begin{eqnarray*}\label{preKL}
   \rho_{KL}((\partial_j)_{p_1},p_2) 
   := (\partial_j)_{p_1}\phi_{KL}(p_1,p_2)
    = - \int_{\Omega}s^{j}(\bm{x},\bm{\theta}_1)p(\bm{x};\bm{\theta}_2)\nu(d\bm{x})
\end{eqnarray*}
%
for any two probability distributions $p_1(\bm{x})=p(\bm{x};\bm{\theta}_1)$,
$p_2(\bm{x})=p(\bm{x};\bm{\theta}_2)$ in $S$ and $j=1,\ldots,d$.
This observation leads to the following proposition \cite{HM}.
%
\begin{proposition}[Pre-contrast Functions from Estimating Functions]
   For an estimating function $\bm{u}(\bm{x},\bm{\theta})$ on the parametric model $S$,
   a pre-contrast function $\rho_{\bms{u}}:TS \times S \rightarrow \bm{R}$ is defined by
%
   \begin{eqnarray}\label{preest}
      \rho_{\bms{u}}((\partial_j)_{p_1},p_2) := 
      - \int_{\Omega}u_{\ast}^j(\bm{x},\bm{\theta}_1)p(\bm{x};\bm{\theta}_2)\nu(d\bm{x})
   \end{eqnarray}
%
   for any two probability distributions $p_1(\bm{x})=p(\bm{x};\bm{\theta}_1)$,
   $p_2(\bm{x})=p(\bm{x};\bm{\theta}_2)$ in $S$ and $j=1,\ldots,d$, where
   $u_{\ast}^j(\bm{x},\bm{\theta})$ is the $j$-th component of the standardization
   $\bm{u}_{\ast}(\bm{x},\bm{\theta})$ of $\bm{u}(\bm{x},\bm{\theta})$.
\end{proposition}
%
The use of the standardization $\bm{u}_{\ast}(\bm{x},\bm{\theta})$ instead of
$\bm{u}(\bm{x},\bm{\theta})$ ensures that the definition of the function $\rho_{\bms{u}}$
does not depend on the choice of coordinate system (parameter) of $S$.
In fact, for a coordinate transformation (parameter transformation) 
$\bm{\eta}=\Phi(\bm{\theta})$, the estimating function $\bm{u}(\bm{x},\bm{\theta})$ is
changed into $\bm{v}(\bm{x},\bm{\eta})=\bm{u}(\bm{x},\Phi^{-1}(\bm{\eta}))$ and we have
%
\begin{eqnarray*}
   \bm{v}_{\ast}(\bm{x},\bm{\eta}) = \left(\frac{\partial\bm{\theta}}{\partial\bm{\eta}}\right)^T
                                     \bm{u}_{\ast}(\bm{x},\bm{\theta}).
\end{eqnarray*}
%
This is the same as the transformation rule of coordinate bases on a tangent space of a
manifold. The set of all components of the standardized estimating function
$\bm{u}_{\ast}(\bm{x},\bm{\theta})$ can be seen as a representation of the coordinate 
basis $\{(\partial_1)_p,\ldots,(\partial_d)_p\}$ on the tangent space $T_p(S)$ of $S$,
where $p(\bm{x})=p(\bm{x};\bm{\theta})$.

The proof of Proposition 2 is straightforward. In particular, the condition $(b)$ in the
definition of pre-contrast function follows from the unbiasedness of the (standardized)
estimating function.
The Riemannian metric $g$, dual connections $\nabla$ and $\nabla^{\ast}$ induced from
the pre-contrast function $\rho_{\bms{u}}$ are given as follows:
%
\begin{eqnarray*}
   &&g_{jk}(\bm{\theta}) := g(\partial_j,\partial_k)
                          = E_{\bms{\theta}}\{u_{\ast}^j(\bm{x},\bm{\theta})
                            u_{\ast}^k(\bm{x},\bm{\theta})\}
                          = G(\bm{\theta})_{jk}, 
                          \label{Godambe} \\
   &&\left\{\begin{array}{l}
             \Gamma_{ij,k}(\bm{\theta}) := g(\nabla_{\partial_i}\partial_j,\partial_k)
             = E_{\bms{\theta}}[\{\partial_iu_{\ast}^j(\bm{x},\bm{\theta})\}s^k(\bm{x},\bm{\theta})] \\
             \Gamma^{\ast}_{ik,j}(\bm{\theta}) 
             := g(\partial_j,\nabla^{\ast}_{\partial_i}\partial_k)
              = \int_{\Omega}u_{\ast}^j(\bm{x},\bm{\theta})\partial_i\partial_kp(\bm{x};\bm{\theta})
                \nu(d\bm{x})
            \end{array}
     \right., \label{dualconn}
\end{eqnarray*}
%
where $G(\bm{\theta})_{jk}$ is the $(j,k)$ component of the Godambe information matrix 
$G(\bm{\theta})$. Note that $\nabla^{\ast}$ is always torsion-free since 
$\Gamma^{\ast}_{ik,j}=\Gamma^{\ast}_{ki,j}$, whereas $\nabla$ is not necessarily 
torsion-free unless $\bm{u}_{\ast}(\bm{x},\bm{\theta})$ is integrable with respect to
$\bm{\theta}$.

If it is integrable and $\nabla$ is torsion-free, it is possible to construct a contrast
function on $S$, from which the pre-contrast function $\rho_{\bms{u}}$ in (\ref{preest})
is obtained by taking its first derivative, as follows:
%
\begin{eqnarray*}
   \phi_{\bms{u}}(p_1,p_2) = \int_{\Omega}\left\{\psi(\bm{x},\bm{\theta}_1) - \psi(\bm{x},\bm{\theta}_2)
                             \right\}p(\bm{x};\bm{\theta}_2)\nu(d\bm{x}),
\end{eqnarray*}
%
where $\partial_j\psi(\bm{x},\bm{\theta})=u_{\ast}^j(\bm{x},\bm{\theta}) \ (j=1,\ldots,d)$
and $p_l(\bm{x})=p(\bm{x};\bm{\theta}_l) \ (l=1,2)$.

%%%%%%%%%%%%%%%%%%%%%%%%%%%%%%%%%%%%%%%%%%%%%%%%%%%%%%%%%%%%%%%%%%%%%%%%%%%%%%%%%%%%%%%%%%%
%%%%%%%% Example %%%%%%%%%%%%%%%%%%%%%%%%%%%%%%%%%%%%%%%%%%%%%%%%%%%%%%%%%%%%%%%%%%%%%%%%%%%
%%%%%%%%%%%%%%%%%%%%%%%%%%%%%%%%%%%%%%%%%%%%%%%%%%%%%%%%%%%%%%%%%%%%%%%%%%%%%%%%%%%%%%%%%%%
\section{Example}
\label{sec:6}
In this section, we consider the estimation problem of voter transition probabilities 
described in \cite{MN} to see an example of statistical manifolds admitting torsion (SMAT) 
induced from estimation functions.

Suppose that we had two successive elections which were carried out in $N$ constituencies,
and that the two political parties C and L contended in each election.
The table below summarizes the numbers of voters in the $n$-th constituency for the respective
elections. It is assumed that we can observe only the marginal totals $m_{1n}, m_{2n}, X_n$
and $m_n-X_n$, where $X_n$ is only a random variable and the others are treated as fixed 
constants.
Let $\theta^1$ and $\theta^2$ be the probabilities that a voter who votes for the parties
C and L in Election $1$ votes for C in Election $2$, respectively. They are the parameters
of interest here. Then, the random variables $X_{1n}$ and $X_{2n}$ in Table $1$ independently
follow the binomial distributions $B(m_{1n},\theta^1)$ and $B(m_{2n},\theta^2)$, respectively.

\begin{table}
\begin{center}
\caption{Votes cast in the $n$-th constituency ($n=1,\ldots,N$)}
\label{tab:1}
\begin{tabular}{p{2cm}p{2cm}p{2cm}p{0.6cm}}
Party & C & L & Total \\ 
\noalign{\smallskip}\hline\noalign{\smallskip}
C & $X_{1n}$ & $m_{1n}-X_{1n}$ & $m_{1n}$ \\
\noalign{\smallskip}\hline\noalign{\smallskip}
L & $X_{2n}$ & $m_{2n}-X_{2n}$ & $m_{2n}$ \\
\noalign{\smallskip}\hline\noalign{\smallskip}
Total & $X_n$ & $m_n-X_n$ & $m_n$ \\
%\noalign{\smallskip}\hline\noalign{\smallskip}
\end{tabular}
\end{center}
\end{table}

\noindent
In the $n$-th constituency, the probability function of the observation $X_n=X_{1n}+X_{2n}$ is given by
%
\begin{eqnarray*}
   p_n(x_n;\bm{\theta}) = \sum_{x_{1n}=0}^{m_{1n}}\left(\begin{array}{c}
                                                           m_{1n} \\
                                                           x_{1n}
                                                        \end{array}
                                                  \right)
                                                  \left(\begin{array}{c}
                                                           m_{2n} \\
                                                           x_n - x_{1n}
                                                        \end{array}
                                                  \right)
                          \left(\theta^1\right)^{x_{1n}}\left(1 - \theta^1\right)^{m_{1n}-x_{1n}}
                          \left(\theta^2\right)^{x_n-x_{1n}}\left(1 - \theta^2\right)^{m_{2n}-x_n+x_{1n}},
\end{eqnarray*}
%
where $\bm{\theta}=\left(\theta^1,\theta^2\right)$.
The statistical model $S$ in this problem consists of all possible probability functions of the observed data
$\bm{X}=(X_1,\ldots,X_N)$ as follows:
%
\begin{eqnarray*}
   S = \left\{p(\bm{x};\bm{\theta}) \left| \ \bm{\theta}=\left(\theta^1,\theta^2\right) \in (0,1) \times (0,1)\right.\right\},
\end{eqnarray*}
%
where $p(\bm{x};\bm{\theta})=\prod_{n=1}^{N}p_n(x_n;\bm{\theta}) \ \left(\bm{x}=(x_1,\ldots,x_n)\right)$ since 
$X_1,\ldots,X_n$ are independent.

Although the maximum likelihood estimation for $\bm{\theta}$ is possible 
based on the likelihood function $L(\bm{\theta})=p(\bm{X};\bm{\theta})$, it is a little complicated since 
$X_{1n}$ and $X_{2n}$ are not observed in each $n$-th constituency. An alternative approach for estimating 
$\bm{\theta}$ is to use the quasi-score function
$\bm{q}(\bm{x},\bm{\theta})=(q^1(\bm{x},\bm{\theta}),q^2(\bm{x},\bm{\theta}))^T$ 
\cite{MN} as an estimating function, where
%
\begin{eqnarray*}\label{conestfun}
   q^1(\bm{x},\bm{\theta}) = \sum_{n=1}^{N}\frac{m_{1n}\{x_n - \mu_n(\bm{\theta})\}}{V_n(\bm{\theta})}, \
   q^2(\bm{x},\bm{\theta}) = \sum_{n=1}^{N}\frac{m_{2n}\{x_n - \mu_n(\bm{\theta})\}}{V_n(\bm{\theta})}.
\end{eqnarray*}
%
Here, $\mu_n(\bm{\theta})$ and $V_n(\bm{\theta})$ are the mean and variance of $X_n$, respectively, i.e.
%
\begin{eqnarray}
   &&\mu_n(\bm{\theta}) = E(X_n) = m_{1n}\theta^1 + m_{2n}\theta^2 \label{mean} \\
   &&V_n(\bm{\theta}) = V(X_n) = m_{1n}\theta^1\left(1-\theta^1\right) + m_{2n}\theta^2\left(1-\theta^2\right). \nonumber
\end{eqnarray}
%
In this example, the random variables $X_1,\ldots,X_n$ in the observed data are independent, but not
identically distributed. However, it is possible to apply the results in Section 5 by considering the
left hand side of (\ref{esteq}) as an estimating function.
Note that the estimating function $\bm{q}(\bm{x},\bm{\theta})$ is already standardized since
the $i$-th component $q^i(\bm{x},\bm{\theta})$ of $\bm{q}(\bm{x},\bm{\theta})$ is obtained
by the orthogonal projection of the $i$-th component of the score function $\bm{s}(\bm{x},\bm{\theta})$ 
for $\bm{\theta}$ onto the linear space spanned by $\{x_1-\mu_1(\bm{\theta}),\ldots,x_N-\mu_N(\bm{\theta})\}$. 
In fact, the orthogonal projection is calculated as follows:
%
\begin{eqnarray*}
   &&E_{\bms{\theta}}\left\{\bm{s}(\bm{x},\bm{\theta})(\bm{x} - \bm{\mu}(\bm{\theta}))^T\right\}
     \left[E_{\bms{\theta}}\left\{(\bm{x} - \bm{\mu}(\bm{\theta}))(\bm{x} - \bm{\mu}(\bm{\theta}))^T\right\}\right]^{-1}
     (\bm{x} - \bm{\mu}(\bm{\theta})) \\
   &=&
      -E_{\bms{\theta}}\left\{\frac{\partial}{\partial\bm{\theta}^T}(\bm{x} - \bm{\mu}(\bm{\theta}))\right\}
      \left[E_{\bms{\theta}}\left\{(\bm{x} - \bm{\mu}(\bm{\theta}))(\bm{x} - \bm{\mu}(\bm{\theta}))^T\right\}\right]^{-1}
      (\bm{x} - \bm{\mu}(\bm{\theta})) \\
   &=&
      \left(\begin{array}{ccc}
               m_{11} & \cdots & m_{1N} \\
               m_{21} & \cdots & m_{2N}
            \end{array}
      \right)
      \left(\begin{array}{ccc}
               V_1(\bm{\theta}) & \cdots & 0 \\
               \vdots               & \ddots & \vdots \\
               0                       & \cdots & V_n(\bm{\theta})
            \end{array}
      \right)
      \left(\begin{array}{c}
               x_1 - \mu_1(\bm{\theta}) \\
               \vdots \\
               x_N - \mu_N(\bm{\theta})
            \end{array}
      \right)
    = \left(\begin{array}{c}
               q^1(\bm{x},\bm{\theta}) \\
               q^2(\bm{x},\bm{\theta})
            \end{array}
      \right),
\end{eqnarray*}
%
where $\bm{x}=(x_1,\ldots,x_N)^T$ and $\bm{\mu}(\bm{\theta})=(\mu_1(\bm{\theta}),\ldots,\mu_N(\bm{\theta}))^T$.
In addition, the estimating function $\bm{q}(\bm{x},\bm{\theta})$ is not integrable with respect
to $\bm{\theta}$ since $\partial q^1/\partial\theta^2 \neq \partial q^2/\partial\theta^1$.
From Proposition 2 and the fact that $\bm{q}(\bm{x},\bm{\theta})$ itself is a standardized estimating function,
we immediately obtain the pre-contrast function $\rho_q:TS \times S \rightarrow \bm{R}$
defined by $\bm{q}(\bm{x},\bm{\theta})$, where
%
\begin{eqnarray*}
   \rho_q((\partial_i)_{p_1},p_2) = -\sum_{\bms{x}}q^i(\bm{x},\bm{\theta}_1)p(\bm{x};\bm{\theta}_2)
                                  = \sum_{n=1}^{N}\frac{m_{in}\{\mu_n(\bm{\theta}_1) - \mu_n(\bm{\theta}_2)\}}
                                    {V_n(\bm{\theta}_1)}
\end{eqnarray*}
%
with $p_l(\bm{x})=p(\bm{x};\bm{\theta}_l) \in S \ (l=1,2)$.
The pre-contrast function $\rho_q$ induces the statistical manifold admitting torsion as follows. \\
%
Riemannian metric $g$:
%
\begin{eqnarray*}
   g_{ij}(\bm{\theta}) = \sum_{\bms{x}}q^i(\bm{x},\bm{\theta})q^j(\bm{x},\bm{\theta})p(\bm{x};\bm{\theta})
                       = \sum_{n=1}^{N}\frac{1}{V_n(\bm{\theta})}m_{in}m_{jn}.
\end{eqnarray*}
%
Dual affine connections $\nabla^{\ast}$ and $\nabla$:
%
\begin{eqnarray*}
   \Gamma_{ij,k}^{\ast}(\bm{\theta}) &=& \sum_{\bms{x}}\{\partial_i\partial_jp(\bm{x};\bm{\theta})\}
                                         q^k(\bm{x},\bm{\theta}) \\
                                     &=& \sum_{n=1}^{N}\frac{m_{kn}}{V_n(\bm{\theta})}
                                         \left[\sum_{\bms{x}}x_n\{\partial_i\partial_jp(\bm{x};\bm{\theta})\}
                                         - \mu_n(\bm{\theta})\sum_{\bms{x}}\{\partial_i\partial_jp(\bm{x};\bm{\theta})\}
                                         \right] \\
                                     &=& \sum_{n=1}^{N}\frac{m_{kn}}{V_n(\bm{\theta})}
                                         \left[\partial_i\partial_j\sum_{\bms{x}}x_np(\bm{x};\bm{\theta})
                                         - \mu_n(\bm{\theta})\partial_i\partial_j\sum_{\bms{x}}p(\bm{x};\bm{\theta})
                                         \right] \\
                                     &=& \sum_{n=1}^{N}\frac{m_{kn}}{V_n(\bm{\theta})}\partial_i\partial_j\mu_n(\bm{\theta})
                                      = 0 \vspace{2mm} \ \ (\mbox{from (\ref{mean})}) \\
   \Gamma_{ij,k}(\bm{\theta}) &=& \Gamma_{ij,k}^{\ast}(\bm{\theta}) - \partial_ig_{jk}(\bm{\theta}) \ \ 
                                  (\mbox{from the duality between $\nabla$ and $\nabla^{\ast}$}) \\
                              &=& \sum_{n=1}^{N}\frac{1-2\theta^i}{V_n(\bm{\theta})^2}m_{in}m_{jn}m_{kn}.
\end{eqnarray*}
%
In this example, the statistical model $S$ is $\nabla^{\ast}$-flat since the coefficient of $\nabla$ 
with respect to the parameter $\bm{\theta}$ is equal to zero.
Furthermore, this shows that $\bm{\theta}$ provides an affine coordinate system for $\nabla$.
Although the curvature tensor of $\nabla$ vanishes because the curvature tensor of $\nabla^{\ast}$
vanishes and $\nabla$ is dual to $\nabla^{\ast}$, the statistical model $S$ is not $\nabla$-flat 
because $\nabla$ is not torsion-free, which comes from the non-integrability of the estimating 
function $\bm{q}(\bm{x},\bm{\theta})$.
Hence, this geometrical structure provides an example of partially flat spaces, which was discussed
in Section 4.

%%%%%%%%%%%%%%%%%%%%%%%%%%%%%%%%%%%%%%%%%%%%%%%%%%%%%%%%%%%%%%%%%%%%%%%%%%%%%%%%%%%%%%%%%%%
%%%%%%%% Futute Problems %%%%%%%%%%%%%%%%%%%%%%%%%%%%%%%%%%%%%%%%%%%%%%%%%%%%%%%%%%%%%%%%%%
%%%%%%%%%%%%%%%%%%%%%%%%%%%%%%%%%%%%%%%%%%%%%%%%%%%%%%%%%%%%%%%%%%%%%%%%%%%%%%%%%%%%%%%%%%%
\section{Future Problems}
\label{sec:6}
In this paper, we have summarized existing results on statistical manifolds admitting
torsion, especially focusing on partially flat spaces. Although some results that are not
seen in the standard theory of information geometry have been obtained, including a
generalized projection theorem in partially flat spaces and statistical manifolds admitting
torsion induced from estimating functions in statistics, a lot of (essential) problems have
been unsolved. We discuss some of them to conclude this paper.
\vspace{3mm} \\
%
(1) The canonical pre-contrast function and the generalized projection theorem in a partially
    flat spase $(M,g,\nabla,\nabla^{\ast})$ are described only in terms of the flat connection
    $\nabla^{\ast}$. In this sense, it can be said that these are a concept and a theorem for
    the Riemannian manifold $(M,g)$ with the flat connection $\nabla^{\ast}$. What is the role
    of the affine connection $\nabla$ that has possibly torsion in the partially flat space
    $(M,g,\nabla,\nabla^{\ast})$? \vspace{2mm} \\
%
(2) The canonical pre-contrast function is defined in terms of the Riemannian metric $g$ and
    the $\nabla^{\ast}$-geodesic in a partially flat space $(U,g,\nabla,\nabla^{\ast})$
    {\it without} using the affine coordinate system $(\eta_i)$ on $U$.
    Hence, this concept can be defined in a general statistical manifold admitting torsion
    $(M,g,\nabla)$ as long as the $\nabla^{\ast}$-geodesic uniquely exists.
    What is the condition under which it is possible? What properties does the canonical 
    pre-contrast function have in this case?
    These problems are closely related to the works by \cite{HK} and \cite{AA}, who try to
    define a canonical divergence (canonical contrast function) on a general statistical
    manifold beyond a dually flat space. \vspace{2mm} \\
%
(3) Relationship between the canonical pre-contrast functions and pre-divergences defined from
    the viewpoint of affine differential geometry. \vspace{2mm} \\
%
(4) The definition of pre-contrast functions from estimating functions is obtained by replacing
    the score function which appears in the pre-contrast function as the derivative of
    Kullback-Leibler divergence with the standardized estimating functions.
    However, this is not the unique way to obtain a pre-contrast function from an estimating
    function. For example, if we consider the $\beta$-divergence \cite{EK} (or density power
    divergence \cite{BHHJ}) as a contrast function, its first derivative is also a pre-contrast
    function and takes the same form as (\ref{preest}) in Proposition 2.
    However, the estimating function which appears in the pre-contrast function is not
    standardized. Although the standardization seems to be natural, further consideration is
    necessary on how to define a pre-contrast function from a given estimating function. \vspace{2mm} \\
%
(5) For the example considered in Section 6,  we can show that the pre-contrast function
     $\rho_{\bms{q}}$ coincides with the canonical pre-contrast function in the partially flat
     space $(S,g,\nabla,\nabla^{\ast})$ and the generalized projection theorem (Corollary 1 
     in Section 4) can be applied. However, its statistical meaning has not been clarified yet. 
     Although it is expected that the SMAT induced from an estimating function has something
     to do with statistical inference based on the estimating function, the clarification on it is 
     a future problem.

\begin{acknowledgement}
This work was supported by JSPS KAKENHI Grant Number 15K00064.
\end{acknowledgement}

\begin{thebibliography}{99.}
%
\bibitem{E}
Eguchi, S.:
Geometry of minimum contrast.
{\em Hiroshima Math. J.} \textbf{22}, 631-647 (1992)

\bibitem{Ma1}
Matsuzoe, H.:
Geometry of contrast functions and conformal geometry.
{\em Hiroshima Math. J.} \textbf{29}, 175-191 (1999)

\bibitem{AN}
Amari, S., Nagaoka, H.:
{\em Method of Information Geometry}.
Amer. Math. Soc., Providence, Oxford University Press, Oxford (2000)

\bibitem{A}
Amari, S.:
{\em Information Geometry and Its Applications}. Springer (2016)

\bibitem{Ku2}
Kurose, T.:
{\em Statistical Manifolds Admitting Torsion}.
Geometry and Something, Fukuoka University (2007)

\bibitem{Ma2}
Matsuzoe, H.:
{\em Statistical Manifolds Admitting Torsion and Pre-contrast Functions}.
Information Geometry and Its Related Fields, Osaka City University (2010)

\bibitem{HM}
Henmi, M., Matsuzoe, H.:
Geometry of pre-contrast functions and non-conservative estimating functions.
{\em AIP Conference Proceedings} \textbf{1340}, 32-41 (2011)

\bibitem{He}
Henmi, M.:
Statistical Manifolds Admitting Torsion, Pre-contrast Functions and Estimating
Functions.
{\em Lecture Notes in Computer Science} \textbf{10589}, 153-161 (2017)

\bibitem{Ku1}
Kurose, T.:
On the divergences of 1-conformally flat statistical manifolds.
{\em Tohoku Math. J.} \textbf{46}, 427-433 (1994)

\bibitem{HK}
Henmi, M., Kobayashi, R.:
Hooke's law in statistical manifolds and divergences.
{\em Nagoya Math. J.} \textbf{159}, 1-24 (2000)

\bibitem{AA}
Ay, N. and Amari, S.:
A novel approach to canonical divergences within information geometry.
{\em Entropy} \textbf{17}, 8111-8129 (2015)

\bibitem{V}
van der Vaart, A.W.:
{\em Asymptotic Statistics}.
Cambridge University Press (2000)

\bibitem{H}
Heyde, C.C.:
{\em Quasi-Likelihood and Its Application}. Springer (1997)

\bibitem{G}
Godambe, V.:
An optimum property of regular maximum likelihood estimation.
{\em Ann. Math. Statist.} \textbf{31}, 1208-1211 (1960)

\bibitem{MN}
McCullagh, P., Nelder J.A.:
{\em Generalized Linear Models} (2nd ed.).
Chapman and Hall (1989)

\bibitem{EK}
Eguchi, S., Kano, Y.:
Robustifying maximum likelihood estimation.
{\em Research Memorandum of the Institute of Statistical Mathematics}
\textbf{802} (2001)

\bibitem{BHHJ}
Basu, A., Harris, I.R., Hjort, N.L., Jones, M.C.:
Robust and efficient estimation by minimizing a density power divergence.
{\em Biometrika} \textbf{85}, 549-559 (1998)
%
\end{thebibliography}
\end{document}
